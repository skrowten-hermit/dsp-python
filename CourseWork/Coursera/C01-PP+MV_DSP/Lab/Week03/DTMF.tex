
% Default to the notebook output style

    


% Inherit from the specified cell style.




    
\documentclass[11pt]{article}

    
    
    \usepackage[T1]{fontenc}
    % Nicer default font (+ math font) than Computer Modern for most use cases
    \usepackage{mathpazo}

    % Basic figure setup, for now with no caption control since it's done
    % automatically by Pandoc (which extracts ![](path) syntax from Markdown).
    \usepackage{graphicx}
    % We will generate all images so they have a width \maxwidth. This means
    % that they will get their normal width if they fit onto the page, but
    % are scaled down if they would overflow the margins.
    \makeatletter
    \def\maxwidth{\ifdim\Gin@nat@width>\linewidth\linewidth
    \else\Gin@nat@width\fi}
    \makeatother
    \let\Oldincludegraphics\includegraphics
    % Set max figure width to be 80% of text width, for now hardcoded.
    \renewcommand{\includegraphics}[1]{\Oldincludegraphics[width=.8\maxwidth]{#1}}
    % Ensure that by default, figures have no caption (until we provide a
    % proper Figure object with a Caption API and a way to capture that
    % in the conversion process - todo).
    \usepackage{caption}
    \DeclareCaptionLabelFormat{nolabel}{}
    \captionsetup{labelformat=nolabel}

    \usepackage{adjustbox} % Used to constrain images to a maximum size 
    \usepackage{xcolor} % Allow colors to be defined
    \usepackage{enumerate} % Needed for markdown enumerations to work
    \usepackage{geometry} % Used to adjust the document margins
    \usepackage{amsmath} % Equations
    \usepackage{amssymb} % Equations
    \usepackage{textcomp} % defines textquotesingle
    % Hack from http://tex.stackexchange.com/a/47451/13684:
    \AtBeginDocument{%
        \def\PYZsq{\textquotesingle}% Upright quotes in Pygmentized code
    }
    \usepackage{upquote} % Upright quotes for verbatim code
    \usepackage{eurosym} % defines \euro
    \usepackage[mathletters]{ucs} % Extended unicode (utf-8) support
    \usepackage[utf8x]{inputenc} % Allow utf-8 characters in the tex document
    \usepackage{fancyvrb} % verbatim replacement that allows latex
    \usepackage{grffile} % extends the file name processing of package graphics 
                         % to support a larger range 
    % The hyperref package gives us a pdf with properly built
    % internal navigation ('pdf bookmarks' for the table of contents,
    % internal cross-reference links, web links for URLs, etc.)
    \usepackage{hyperref}
    \usepackage{longtable} % longtable support required by pandoc >1.10
    \usepackage{booktabs}  % table support for pandoc > 1.12.2
    \usepackage[inline]{enumitem} % IRkernel/repr support (it uses the enumerate* environment)
    \usepackage[normalem]{ulem} % ulem is needed to support strikethroughs (\sout)
                                % normalem makes italics be italics, not underlines
    

    
    
    % Colors for the hyperref package
    \definecolor{urlcolor}{rgb}{0,.145,.698}
    \definecolor{linkcolor}{rgb}{.71,0.21,0.01}
    \definecolor{citecolor}{rgb}{.12,.54,.11}

    % ANSI colors
    \definecolor{ansi-black}{HTML}{3E424D}
    \definecolor{ansi-black-intense}{HTML}{282C36}
    \definecolor{ansi-red}{HTML}{E75C58}
    \definecolor{ansi-red-intense}{HTML}{B22B31}
    \definecolor{ansi-green}{HTML}{00A250}
    \definecolor{ansi-green-intense}{HTML}{007427}
    \definecolor{ansi-yellow}{HTML}{DDB62B}
    \definecolor{ansi-yellow-intense}{HTML}{B27D12}
    \definecolor{ansi-blue}{HTML}{208FFB}
    \definecolor{ansi-blue-intense}{HTML}{0065CA}
    \definecolor{ansi-magenta}{HTML}{D160C4}
    \definecolor{ansi-magenta-intense}{HTML}{A03196}
    \definecolor{ansi-cyan}{HTML}{60C6C8}
    \definecolor{ansi-cyan-intense}{HTML}{258F8F}
    \definecolor{ansi-white}{HTML}{C5C1B4}
    \definecolor{ansi-white-intense}{HTML}{A1A6B2}

    % commands and environments needed by pandoc snippets
    % extracted from the output of `pandoc -s`
    \providecommand{\tightlist}{%
      \setlength{\itemsep}{0pt}\setlength{\parskip}{0pt}}
    \DefineVerbatimEnvironment{Highlighting}{Verbatim}{commandchars=\\\{\}}
    % Add ',fontsize=\small' for more characters per line
    \newenvironment{Shaded}{}{}
    \newcommand{\KeywordTok}[1]{\textcolor[rgb]{0.00,0.44,0.13}{\textbf{{#1}}}}
    \newcommand{\DataTypeTok}[1]{\textcolor[rgb]{0.56,0.13,0.00}{{#1}}}
    \newcommand{\DecValTok}[1]{\textcolor[rgb]{0.25,0.63,0.44}{{#1}}}
    \newcommand{\BaseNTok}[1]{\textcolor[rgb]{0.25,0.63,0.44}{{#1}}}
    \newcommand{\FloatTok}[1]{\textcolor[rgb]{0.25,0.63,0.44}{{#1}}}
    \newcommand{\CharTok}[1]{\textcolor[rgb]{0.25,0.44,0.63}{{#1}}}
    \newcommand{\StringTok}[1]{\textcolor[rgb]{0.25,0.44,0.63}{{#1}}}
    \newcommand{\CommentTok}[1]{\textcolor[rgb]{0.38,0.63,0.69}{\textit{{#1}}}}
    \newcommand{\OtherTok}[1]{\textcolor[rgb]{0.00,0.44,0.13}{{#1}}}
    \newcommand{\AlertTok}[1]{\textcolor[rgb]{1.00,0.00,0.00}{\textbf{{#1}}}}
    \newcommand{\FunctionTok}[1]{\textcolor[rgb]{0.02,0.16,0.49}{{#1}}}
    \newcommand{\RegionMarkerTok}[1]{{#1}}
    \newcommand{\ErrorTok}[1]{\textcolor[rgb]{1.00,0.00,0.00}{\textbf{{#1}}}}
    \newcommand{\NormalTok}[1]{{#1}}
    
    % Additional commands for more recent versions of Pandoc
    \newcommand{\ConstantTok}[1]{\textcolor[rgb]{0.53,0.00,0.00}{{#1}}}
    \newcommand{\SpecialCharTok}[1]{\textcolor[rgb]{0.25,0.44,0.63}{{#1}}}
    \newcommand{\VerbatimStringTok}[1]{\textcolor[rgb]{0.25,0.44,0.63}{{#1}}}
    \newcommand{\SpecialStringTok}[1]{\textcolor[rgb]{0.73,0.40,0.53}{{#1}}}
    \newcommand{\ImportTok}[1]{{#1}}
    \newcommand{\DocumentationTok}[1]{\textcolor[rgb]{0.73,0.13,0.13}{\textit{{#1}}}}
    \newcommand{\AnnotationTok}[1]{\textcolor[rgb]{0.38,0.63,0.69}{\textbf{\textit{{#1}}}}}
    \newcommand{\CommentVarTok}[1]{\textcolor[rgb]{0.38,0.63,0.69}{\textbf{\textit{{#1}}}}}
    \newcommand{\VariableTok}[1]{\textcolor[rgb]{0.10,0.09,0.49}{{#1}}}
    \newcommand{\ControlFlowTok}[1]{\textcolor[rgb]{0.00,0.44,0.13}{\textbf{{#1}}}}
    \newcommand{\OperatorTok}[1]{\textcolor[rgb]{0.40,0.40,0.40}{{#1}}}
    \newcommand{\BuiltInTok}[1]{{#1}}
    \newcommand{\ExtensionTok}[1]{{#1}}
    \newcommand{\PreprocessorTok}[1]{\textcolor[rgb]{0.74,0.48,0.00}{{#1}}}
    \newcommand{\AttributeTok}[1]{\textcolor[rgb]{0.49,0.56,0.16}{{#1}}}
    \newcommand{\InformationTok}[1]{\textcolor[rgb]{0.38,0.63,0.69}{\textbf{\textit{{#1}}}}}
    \newcommand{\WarningTok}[1]{\textcolor[rgb]{0.38,0.63,0.69}{\textbf{\textit{{#1}}}}}
    
    
    % Define a nice break command that doesn't care if a line doesn't already
    % exist.
    \def\br{\hspace*{\fill} \\* }
    % Math Jax compatability definitions
    \def\gt{>}
    \def\lt{<}
    % Document parameters
    \title{DTMF}
    
    
    

    % Pygments definitions
    
\makeatletter
\def\PY@reset{\let\PY@it=\relax \let\PY@bf=\relax%
    \let\PY@ul=\relax \let\PY@tc=\relax%
    \let\PY@bc=\relax \let\PY@ff=\relax}
\def\PY@tok#1{\csname PY@tok@#1\endcsname}
\def\PY@toks#1+{\ifx\relax#1\empty\else%
    \PY@tok{#1}\expandafter\PY@toks\fi}
\def\PY@do#1{\PY@bc{\PY@tc{\PY@ul{%
    \PY@it{\PY@bf{\PY@ff{#1}}}}}}}
\def\PY#1#2{\PY@reset\PY@toks#1+\relax+\PY@do{#2}}

\expandafter\def\csname PY@tok@w\endcsname{\def\PY@tc##1{\textcolor[rgb]{0.73,0.73,0.73}{##1}}}
\expandafter\def\csname PY@tok@c\endcsname{\let\PY@it=\textit\def\PY@tc##1{\textcolor[rgb]{0.25,0.50,0.50}{##1}}}
\expandafter\def\csname PY@tok@cp\endcsname{\def\PY@tc##1{\textcolor[rgb]{0.74,0.48,0.00}{##1}}}
\expandafter\def\csname PY@tok@k\endcsname{\let\PY@bf=\textbf\def\PY@tc##1{\textcolor[rgb]{0.00,0.50,0.00}{##1}}}
\expandafter\def\csname PY@tok@kp\endcsname{\def\PY@tc##1{\textcolor[rgb]{0.00,0.50,0.00}{##1}}}
\expandafter\def\csname PY@tok@kt\endcsname{\def\PY@tc##1{\textcolor[rgb]{0.69,0.00,0.25}{##1}}}
\expandafter\def\csname PY@tok@o\endcsname{\def\PY@tc##1{\textcolor[rgb]{0.40,0.40,0.40}{##1}}}
\expandafter\def\csname PY@tok@ow\endcsname{\let\PY@bf=\textbf\def\PY@tc##1{\textcolor[rgb]{0.67,0.13,1.00}{##1}}}
\expandafter\def\csname PY@tok@nb\endcsname{\def\PY@tc##1{\textcolor[rgb]{0.00,0.50,0.00}{##1}}}
\expandafter\def\csname PY@tok@nf\endcsname{\def\PY@tc##1{\textcolor[rgb]{0.00,0.00,1.00}{##1}}}
\expandafter\def\csname PY@tok@nc\endcsname{\let\PY@bf=\textbf\def\PY@tc##1{\textcolor[rgb]{0.00,0.00,1.00}{##1}}}
\expandafter\def\csname PY@tok@nn\endcsname{\let\PY@bf=\textbf\def\PY@tc##1{\textcolor[rgb]{0.00,0.00,1.00}{##1}}}
\expandafter\def\csname PY@tok@ne\endcsname{\let\PY@bf=\textbf\def\PY@tc##1{\textcolor[rgb]{0.82,0.25,0.23}{##1}}}
\expandafter\def\csname PY@tok@nv\endcsname{\def\PY@tc##1{\textcolor[rgb]{0.10,0.09,0.49}{##1}}}
\expandafter\def\csname PY@tok@no\endcsname{\def\PY@tc##1{\textcolor[rgb]{0.53,0.00,0.00}{##1}}}
\expandafter\def\csname PY@tok@nl\endcsname{\def\PY@tc##1{\textcolor[rgb]{0.63,0.63,0.00}{##1}}}
\expandafter\def\csname PY@tok@ni\endcsname{\let\PY@bf=\textbf\def\PY@tc##1{\textcolor[rgb]{0.60,0.60,0.60}{##1}}}
\expandafter\def\csname PY@tok@na\endcsname{\def\PY@tc##1{\textcolor[rgb]{0.49,0.56,0.16}{##1}}}
\expandafter\def\csname PY@tok@nt\endcsname{\let\PY@bf=\textbf\def\PY@tc##1{\textcolor[rgb]{0.00,0.50,0.00}{##1}}}
\expandafter\def\csname PY@tok@nd\endcsname{\def\PY@tc##1{\textcolor[rgb]{0.67,0.13,1.00}{##1}}}
\expandafter\def\csname PY@tok@s\endcsname{\def\PY@tc##1{\textcolor[rgb]{0.73,0.13,0.13}{##1}}}
\expandafter\def\csname PY@tok@sd\endcsname{\let\PY@it=\textit\def\PY@tc##1{\textcolor[rgb]{0.73,0.13,0.13}{##1}}}
\expandafter\def\csname PY@tok@si\endcsname{\let\PY@bf=\textbf\def\PY@tc##1{\textcolor[rgb]{0.73,0.40,0.53}{##1}}}
\expandafter\def\csname PY@tok@se\endcsname{\let\PY@bf=\textbf\def\PY@tc##1{\textcolor[rgb]{0.73,0.40,0.13}{##1}}}
\expandafter\def\csname PY@tok@sr\endcsname{\def\PY@tc##1{\textcolor[rgb]{0.73,0.40,0.53}{##1}}}
\expandafter\def\csname PY@tok@ss\endcsname{\def\PY@tc##1{\textcolor[rgb]{0.10,0.09,0.49}{##1}}}
\expandafter\def\csname PY@tok@sx\endcsname{\def\PY@tc##1{\textcolor[rgb]{0.00,0.50,0.00}{##1}}}
\expandafter\def\csname PY@tok@m\endcsname{\def\PY@tc##1{\textcolor[rgb]{0.40,0.40,0.40}{##1}}}
\expandafter\def\csname PY@tok@gh\endcsname{\let\PY@bf=\textbf\def\PY@tc##1{\textcolor[rgb]{0.00,0.00,0.50}{##1}}}
\expandafter\def\csname PY@tok@gu\endcsname{\let\PY@bf=\textbf\def\PY@tc##1{\textcolor[rgb]{0.50,0.00,0.50}{##1}}}
\expandafter\def\csname PY@tok@gd\endcsname{\def\PY@tc##1{\textcolor[rgb]{0.63,0.00,0.00}{##1}}}
\expandafter\def\csname PY@tok@gi\endcsname{\def\PY@tc##1{\textcolor[rgb]{0.00,0.63,0.00}{##1}}}
\expandafter\def\csname PY@tok@gr\endcsname{\def\PY@tc##1{\textcolor[rgb]{1.00,0.00,0.00}{##1}}}
\expandafter\def\csname PY@tok@ge\endcsname{\let\PY@it=\textit}
\expandafter\def\csname PY@tok@gs\endcsname{\let\PY@bf=\textbf}
\expandafter\def\csname PY@tok@gp\endcsname{\let\PY@bf=\textbf\def\PY@tc##1{\textcolor[rgb]{0.00,0.00,0.50}{##1}}}
\expandafter\def\csname PY@tok@go\endcsname{\def\PY@tc##1{\textcolor[rgb]{0.53,0.53,0.53}{##1}}}
\expandafter\def\csname PY@tok@gt\endcsname{\def\PY@tc##1{\textcolor[rgb]{0.00,0.27,0.87}{##1}}}
\expandafter\def\csname PY@tok@err\endcsname{\def\PY@bc##1{\setlength{\fboxsep}{0pt}\fcolorbox[rgb]{1.00,0.00,0.00}{1,1,1}{\strut ##1}}}
\expandafter\def\csname PY@tok@kc\endcsname{\let\PY@bf=\textbf\def\PY@tc##1{\textcolor[rgb]{0.00,0.50,0.00}{##1}}}
\expandafter\def\csname PY@tok@kd\endcsname{\let\PY@bf=\textbf\def\PY@tc##1{\textcolor[rgb]{0.00,0.50,0.00}{##1}}}
\expandafter\def\csname PY@tok@kn\endcsname{\let\PY@bf=\textbf\def\PY@tc##1{\textcolor[rgb]{0.00,0.50,0.00}{##1}}}
\expandafter\def\csname PY@tok@kr\endcsname{\let\PY@bf=\textbf\def\PY@tc##1{\textcolor[rgb]{0.00,0.50,0.00}{##1}}}
\expandafter\def\csname PY@tok@bp\endcsname{\def\PY@tc##1{\textcolor[rgb]{0.00,0.50,0.00}{##1}}}
\expandafter\def\csname PY@tok@fm\endcsname{\def\PY@tc##1{\textcolor[rgb]{0.00,0.00,1.00}{##1}}}
\expandafter\def\csname PY@tok@vc\endcsname{\def\PY@tc##1{\textcolor[rgb]{0.10,0.09,0.49}{##1}}}
\expandafter\def\csname PY@tok@vg\endcsname{\def\PY@tc##1{\textcolor[rgb]{0.10,0.09,0.49}{##1}}}
\expandafter\def\csname PY@tok@vi\endcsname{\def\PY@tc##1{\textcolor[rgb]{0.10,0.09,0.49}{##1}}}
\expandafter\def\csname PY@tok@vm\endcsname{\def\PY@tc##1{\textcolor[rgb]{0.10,0.09,0.49}{##1}}}
\expandafter\def\csname PY@tok@sa\endcsname{\def\PY@tc##1{\textcolor[rgb]{0.73,0.13,0.13}{##1}}}
\expandafter\def\csname PY@tok@sb\endcsname{\def\PY@tc##1{\textcolor[rgb]{0.73,0.13,0.13}{##1}}}
\expandafter\def\csname PY@tok@sc\endcsname{\def\PY@tc##1{\textcolor[rgb]{0.73,0.13,0.13}{##1}}}
\expandafter\def\csname PY@tok@dl\endcsname{\def\PY@tc##1{\textcolor[rgb]{0.73,0.13,0.13}{##1}}}
\expandafter\def\csname PY@tok@s2\endcsname{\def\PY@tc##1{\textcolor[rgb]{0.73,0.13,0.13}{##1}}}
\expandafter\def\csname PY@tok@sh\endcsname{\def\PY@tc##1{\textcolor[rgb]{0.73,0.13,0.13}{##1}}}
\expandafter\def\csname PY@tok@s1\endcsname{\def\PY@tc##1{\textcolor[rgb]{0.73,0.13,0.13}{##1}}}
\expandafter\def\csname PY@tok@mb\endcsname{\def\PY@tc##1{\textcolor[rgb]{0.40,0.40,0.40}{##1}}}
\expandafter\def\csname PY@tok@mf\endcsname{\def\PY@tc##1{\textcolor[rgb]{0.40,0.40,0.40}{##1}}}
\expandafter\def\csname PY@tok@mh\endcsname{\def\PY@tc##1{\textcolor[rgb]{0.40,0.40,0.40}{##1}}}
\expandafter\def\csname PY@tok@mi\endcsname{\def\PY@tc##1{\textcolor[rgb]{0.40,0.40,0.40}{##1}}}
\expandafter\def\csname PY@tok@il\endcsname{\def\PY@tc##1{\textcolor[rgb]{0.40,0.40,0.40}{##1}}}
\expandafter\def\csname PY@tok@mo\endcsname{\def\PY@tc##1{\textcolor[rgb]{0.40,0.40,0.40}{##1}}}
\expandafter\def\csname PY@tok@ch\endcsname{\let\PY@it=\textit\def\PY@tc##1{\textcolor[rgb]{0.25,0.50,0.50}{##1}}}
\expandafter\def\csname PY@tok@cm\endcsname{\let\PY@it=\textit\def\PY@tc##1{\textcolor[rgb]{0.25,0.50,0.50}{##1}}}
\expandafter\def\csname PY@tok@cpf\endcsname{\let\PY@it=\textit\def\PY@tc##1{\textcolor[rgb]{0.25,0.50,0.50}{##1}}}
\expandafter\def\csname PY@tok@c1\endcsname{\let\PY@it=\textit\def\PY@tc##1{\textcolor[rgb]{0.25,0.50,0.50}{##1}}}
\expandafter\def\csname PY@tok@cs\endcsname{\let\PY@it=\textit\def\PY@tc##1{\textcolor[rgb]{0.25,0.50,0.50}{##1}}}

\def\PYZbs{\char`\\}
\def\PYZus{\char`\_}
\def\PYZob{\char`\{}
\def\PYZcb{\char`\}}
\def\PYZca{\char`\^}
\def\PYZam{\char`\&}
\def\PYZlt{\char`\<}
\def\PYZgt{\char`\>}
\def\PYZsh{\char`\#}
\def\PYZpc{\char`\%}
\def\PYZdl{\char`\$}
\def\PYZhy{\char`\-}
\def\PYZsq{\char`\'}
\def\PYZdq{\char`\"}
\def\PYZti{\char`\~}
% for compatibility with earlier versions
\def\PYZat{@}
\def\PYZlb{[}
\def\PYZrb{]}
\makeatother


    % Exact colors from NB
    \definecolor{incolor}{rgb}{0.0, 0.0, 0.5}
    \definecolor{outcolor}{rgb}{0.545, 0.0, 0.0}



    
    % Prevent overflowing lines due to hard-to-break entities
    \sloppy 
    % Setup hyperref package
    \hypersetup{
      breaklinks=true,  % so long urls are correctly broken across lines
      colorlinks=true,
      urlcolor=urlcolor,
      linkcolor=linkcolor,
      citecolor=citecolor,
      }
    % Slightly bigger margins than the latex defaults
    
    \geometry{verbose,tmargin=1in,bmargin=1in,lmargin=1in,rmargin=1in}
    
    

    \begin{document}
    
    
    \maketitle
    
    

    
    \section{Dual-tone multi-frequency (DTMF)
signaling}\label{dual-tone-multi-frequency-dtmf-signaling}

DTMF signaling is the way analog phones send the number dialed by a user
over to the central phone office. This was in the day before all-digital
networks and cell phones were the norm, but the method is still used for
in-call option selection ("press 4 to talk to customer service"...).

The mechanism is rather clever: the phone's keypad is arranged in a
\(4\times 3\) grid and each button is associated to \emph{two}
frequencies according to this table:

\begin{longtable}[]{@{}lccc@{}}
\toprule
& \textbf{1209 Hz} & \textbf{1336 Hz} & \textbf{1477 Hz}\tabularnewline
\midrule
\endhead
\textbf{697 Hz} & 1 & 2 & 3\tabularnewline
\textbf{770 Hz} & 4 & 5 & 6\tabularnewline
\textbf{852 Hz} & 7 & 8 & 9\tabularnewline
\textbf{941 Hz} & * & 0 & \#\tabularnewline
\bottomrule
\end{longtable}

The frequencies in the table have been chosen so that they are
"coprime"; in other words, no frequency is a multiple of any other,
which reduces the probability of erroneously detecting the received
signals due to interference. When a button is pressed, the two
corresponding frequencies are generated simultaneously and sent over the
line. For instance, if the digit '1' is pressed, the generated signal
will be:

\[
    x(t) = \sin(2\pi\cdot 1209\cdot t) + \sin(2\pi\cdot697\cdot t)
\]

The official specifications for the DTMF standard further stipulate
that:

\begin{itemize}
\tightlist
\item
  each tone should be at least 65ms long
\item
  tones corresponding to successive digits should be separated by a
  silent gap of at least 65ms
\end{itemize}

In this notebook we will build a DTMF decoder based on the Discrete
Fourier Transform. Of course here we will use discrete-time signals
exclusively so, if the clock of the system is \(F_s\), each DTMF tone
will be of the form: \[
    x[n] = \sin(2\pi\,(f_l/F_s)\, n) + \sin(2\pi\,(f_h/F_s)\,n)
\]

The first thing to do is to write a DTMF encoder.

    \begin{Verbatim}[commandchars=\\\{\}]
{\color{incolor}In [{\color{incolor}1}]:} \PY{c+c1}{\PYZsh{} first our usual bookkeeping}
        \PY{o}{\PYZpc{}}\PY{k}{pylab} inline
        \PY{k+kn}{import} \PY{n+nn}{matplotlib}
        \PY{k+kn}{import} \PY{n+nn}{matplotlib}\PY{n+nn}{.}\PY{n+nn}{pyplot} \PY{k}{as} \PY{n+nn}{plt}
        \PY{k+kn}{import} \PY{n+nn}{numpy} \PY{k}{as} \PY{n+nn}{np}
        \PY{k+kn}{import} \PY{n+nn}{IPython}
        
        \PY{c+c1}{\PYZsh{} the \PYZdq{}clock\PYZdq{} of the system}
        \PY{n}{FS} \PY{o}{=} \PY{l+m+mi}{24000}
\end{Verbatim}


    \begin{Verbatim}[commandchars=\\\{\}]
Populating the interactive namespace from numpy and matplotlib

    \end{Verbatim}

    \begin{Verbatim}[commandchars=\\\{\}]
{\color{incolor}In [{\color{incolor}2}]:} \PY{k}{def} \PY{n+nf}{dtmf\PYZus{}dial}\PY{p}{(}\PY{n}{number}\PY{p}{)}\PY{p}{:}
            \PY{n}{DTMF} \PY{o}{=} \PY{p}{\PYZob{}}
                \PY{l+s+s1}{\PYZsq{}}\PY{l+s+s1}{1}\PY{l+s+s1}{\PYZsq{}}\PY{p}{:} \PY{p}{(}\PY{l+m+mi}{697}\PY{p}{,} \PY{l+m+mi}{1209}\PY{p}{)}\PY{p}{,} \PY{l+s+s1}{\PYZsq{}}\PY{l+s+s1}{2}\PY{l+s+s1}{\PYZsq{}}\PY{p}{:} \PY{p}{(}\PY{l+m+mi}{697}\PY{p}{,} \PY{l+m+mi}{1336}\PY{p}{)}\PY{p}{,} \PY{l+s+s1}{\PYZsq{}}\PY{l+s+s1}{3}\PY{l+s+s1}{\PYZsq{}}\PY{p}{:} \PY{p}{(}\PY{l+m+mi}{697}\PY{p}{,} \PY{l+m+mi}{1477}\PY{p}{)}\PY{p}{,}
                \PY{l+s+s1}{\PYZsq{}}\PY{l+s+s1}{4}\PY{l+s+s1}{\PYZsq{}}\PY{p}{:} \PY{p}{(}\PY{l+m+mi}{770}\PY{p}{,} \PY{l+m+mi}{1209}\PY{p}{)}\PY{p}{,} \PY{l+s+s1}{\PYZsq{}}\PY{l+s+s1}{5}\PY{l+s+s1}{\PYZsq{}}\PY{p}{:} \PY{p}{(}\PY{l+m+mi}{770}\PY{p}{,} \PY{l+m+mi}{1336}\PY{p}{)}\PY{p}{,} \PY{l+s+s1}{\PYZsq{}}\PY{l+s+s1}{6}\PY{l+s+s1}{\PYZsq{}}\PY{p}{:} \PY{p}{(}\PY{l+m+mi}{770}\PY{p}{,} \PY{l+m+mi}{1477}\PY{p}{)}\PY{p}{,}
                \PY{l+s+s1}{\PYZsq{}}\PY{l+s+s1}{7}\PY{l+s+s1}{\PYZsq{}}\PY{p}{:} \PY{p}{(}\PY{l+m+mi}{852}\PY{p}{,} \PY{l+m+mi}{1209}\PY{p}{)}\PY{p}{,} \PY{l+s+s1}{\PYZsq{}}\PY{l+s+s1}{8}\PY{l+s+s1}{\PYZsq{}}\PY{p}{:} \PY{p}{(}\PY{l+m+mi}{852}\PY{p}{,} \PY{l+m+mi}{1336}\PY{p}{)}\PY{p}{,} \PY{l+s+s1}{\PYZsq{}}\PY{l+s+s1}{9}\PY{l+s+s1}{\PYZsq{}}\PY{p}{:} \PY{p}{(}\PY{l+m+mi}{852}\PY{p}{,} \PY{l+m+mi}{1477}\PY{p}{)}\PY{p}{,}
                \PY{l+s+s1}{\PYZsq{}}\PY{l+s+s1}{*}\PY{l+s+s1}{\PYZsq{}}\PY{p}{:} \PY{p}{(}\PY{l+m+mi}{941}\PY{p}{,} \PY{l+m+mi}{1209}\PY{p}{)}\PY{p}{,} \PY{l+s+s1}{\PYZsq{}}\PY{l+s+s1}{0}\PY{l+s+s1}{\PYZsq{}}\PY{p}{:} \PY{p}{(}\PY{l+m+mi}{941}\PY{p}{,} \PY{l+m+mi}{1336}\PY{p}{)}\PY{p}{,} \PY{l+s+s1}{\PYZsq{}}\PY{l+s+s1}{\PYZsh{}}\PY{l+s+s1}{\PYZsq{}}\PY{p}{:} \PY{p}{(}\PY{l+m+mi}{941}\PY{p}{,} \PY{l+m+mi}{1477}\PY{p}{)}\PY{p}{,}        
            \PY{p}{\PYZcb{}}
            \PY{n}{MARK} \PY{o}{=} \PY{l+m+mf}{0.1}
            \PY{n}{SPACE} \PY{o}{=} \PY{l+m+mf}{0.1}
            \PY{n}{n} \PY{o}{=} \PY{n}{np}\PY{o}{.}\PY{n}{arange}\PY{p}{(}\PY{l+m+mi}{0}\PY{p}{,} \PY{n+nb}{int}\PY{p}{(}\PY{n}{MARK} \PY{o}{*} \PY{n}{FS}\PY{p}{)}\PY{p}{)}
            \PY{n}{x} \PY{o}{=} \PY{n}{np}\PY{o}{.}\PY{n}{array}\PY{p}{(}\PY{p}{[}\PY{p}{]}\PY{p}{)}
            \PY{k}{for} \PY{n}{d} \PY{o+ow}{in} \PY{n}{number}\PY{p}{:}
                \PY{n}{s} \PY{o}{=} \PY{n}{np}\PY{o}{.}\PY{n}{sin}\PY{p}{(}\PY{l+m+mi}{2}\PY{o}{*}\PY{n}{np}\PY{o}{.}\PY{n}{pi} \PY{o}{*} \PY{n}{DTMF}\PY{p}{[}\PY{n}{d}\PY{p}{]}\PY{p}{[}\PY{l+m+mi}{0}\PY{p}{]} \PY{o}{/} \PY{n}{FS} \PY{o}{*} \PY{n}{n}\PY{p}{)} \PY{o}{+} \PY{n}{np}\PY{o}{.}\PY{n}{sin}\PY{p}{(}\PY{l+m+mi}{2}\PY{o}{*}\PY{n}{np}\PY{o}{.}\PY{n}{pi} \PY{o}{*} \PY{n}{DTMF}\PY{p}{[}\PY{n}{d}\PY{p}{]}\PY{p}{[}\PY{l+m+mi}{1}\PY{p}{]} \PY{o}{/} \PY{n}{FS} \PY{o}{*} \PY{n}{n}\PY{p}{)} 
                \PY{n}{x} \PY{o}{=} \PY{n}{np}\PY{o}{.}\PY{n}{concatenate}\PY{p}{(}\PY{p}{(}\PY{n}{x}\PY{p}{,} \PY{n}{s}\PY{p}{,} \PY{n}{np}\PY{o}{.}\PY{n}{zeros}\PY{p}{(}\PY{n+nb}{int}\PY{p}{(}\PY{n}{SPACE} \PY{o}{*} \PY{n}{FS}\PY{p}{)}\PY{p}{)}\PY{p}{)}\PY{p}{)}
            \PY{k}{return} \PY{n}{x}
\end{Verbatim}


    OK, that was easy. Let's test it and evaluate it "by ear":

    \begin{Verbatim}[commandchars=\\\{\}]
{\color{incolor}In [{\color{incolor}3}]:} \PY{n}{x}\PY{o}{=}\PY{n}{dtmf\PYZus{}dial}\PY{p}{(}\PY{l+s+s1}{\PYZsq{}}\PY{l+s+s1}{123\PYZsh{}\PYZsh{}45}\PY{l+s+s1}{\PYZsq{}}\PY{p}{)}
        
        \PY{n}{IPython}\PY{o}{.}\PY{n}{display}\PY{o}{.}\PY{n}{Audio}\PY{p}{(}\PY{n}{x}\PY{p}{,} \PY{n}{rate}\PY{o}{=}\PY{n}{FS}\PY{p}{)}
\end{Verbatim}


\begin{Verbatim}[commandchars=\\\{\}]
{\color{outcolor}Out[{\color{outcolor}3}]:} <IPython.lib.display.Audio object>
\end{Verbatim}
            
    Now let's start thinking about the decoder. We will use the following
strategy:

\begin{itemize}
\tightlist
\item
  split the signal into individual digit tones by looking at the
  position of the gaps
\item
  perform a DFT on the digit tones
\item
  look at the peaks of the Fourier transform and recover the dialed
  number
\end{itemize}

Here we assume whe have the whole signal in memory, i.e. we will perform
\emph{batch} processing; clearly a more practical system would decode
the incoming signal as it arrives sample by sample (real-time
processing); you are more than encouraged to try and implement such an
algorithm.

To split the signal the idea is to look at the local energy over small
windows: when the signal is silence, we will cut it.

Let's see how we can do that; let's look at the raw data first

    \begin{Verbatim}[commandchars=\\\{\}]
{\color{incolor}In [{\color{incolor}4}]:} \PY{n}{plt}\PY{o}{.}\PY{n}{plot}\PY{p}{(}\PY{n}{x}\PY{p}{)}\PY{p}{;}
\end{Verbatim}


    \begin{center}
    \adjustimage{max size={0.9\linewidth}{0.9\paperheight}}{output_6_0.png}
    \end{center}
    { \hspace*{\fill} \\}
    
    OK so, clearly, we should be able to find the high and low energy
sections of the signal. Let's say that we use an analysis window of 240
samples which, at our \(F_s\) corresponds to an interval of 10ms. We can
easily find the local energy like so:

    \begin{Verbatim}[commandchars=\\\{\}]
{\color{incolor}In [{\color{incolor}5}]:} \PY{c+c1}{\PYZsh{} split the signal in 240\PYZhy{}sample chunks and arrange them as rows in a matrix}
        \PY{c+c1}{\PYZsh{} (truncate the data vector to a length multiple of 240 to avoid errors)}
        \PY{n}{w} \PY{o}{=} \PY{n}{np}\PY{o}{.}\PY{n}{reshape}\PY{p}{(}\PY{n}{x}\PY{p}{[}\PY{p}{:}\PY{p}{(}\PY{n+nb}{len}\PY{p}{(}\PY{n}{x}\PY{p}{)}\PY{o}{/}\PY{l+m+mi}{240}\PY{p}{)}\PY{o}{*}\PY{l+m+mi}{240}\PY{p}{]}\PY{p}{,} \PY{p}{(}\PY{o}{\PYZhy{}}\PY{l+m+mi}{1}\PY{p}{,} \PY{l+m+mi}{240}\PY{p}{)}\PY{p}{)}
        
        \PY{c+c1}{\PYZsh{} compute the energy of each chunk by summing the squares of the elements of each row}
        \PY{n}{we} \PY{o}{=} \PY{n}{np}\PY{o}{.}\PY{n}{sum}\PY{p}{(}\PY{n}{w} \PY{o}{*} \PY{n}{w}\PY{p}{,} \PY{n}{axis}\PY{o}{=}\PY{l+m+mi}{1}\PY{p}{)}
        
        \PY{n}{plt}\PY{o}{.}\PY{n}{plot}\PY{p}{(}\PY{n}{we}\PY{p}{)}\PY{p}{;}
\end{Verbatim}


    \begin{center}
    \adjustimage{max size={0.9\linewidth}{0.9\paperheight}}{output_8_0.png}
    \end{center}
    { \hspace*{\fill} \\}
    
    From the plot, it appears clearly that we can set a threshold of about
200 to separate tone sections from silence sections. Let's write a
function that returns the start and stop indices of the tone sections in
an input signal

    \begin{Verbatim}[commandchars=\\\{\}]
{\color{incolor}In [{\color{incolor}6}]:} \PY{k}{def} \PY{n+nf}{dtmf\PYZus{}split}\PY{p}{(}\PY{n}{x}\PY{p}{,} \PY{n}{win}\PY{o}{=}\PY{l+m+mi}{240}\PY{p}{,} \PY{n}{th}\PY{o}{=}\PY{l+m+mi}{200}\PY{p}{)}\PY{p}{:}
            \PY{n}{edges} \PY{o}{=} \PY{p}{[}\PY{p}{]}
            
            \PY{n}{w} \PY{o}{=} \PY{n}{np}\PY{o}{.}\PY{n}{reshape}\PY{p}{(}\PY{n}{x}\PY{p}{[}\PY{p}{:}\PY{p}{(}\PY{n+nb}{len}\PY{p}{(}\PY{n}{x}\PY{p}{)}\PY{o}{/}\PY{n}{win}\PY{p}{)}\PY{o}{*}\PY{n}{win}\PY{p}{]}\PY{p}{,} \PY{p}{(}\PY{o}{\PYZhy{}}\PY{l+m+mi}{1}\PY{p}{,} \PY{n}{win}\PY{p}{)}\PY{p}{)}
            \PY{n}{we} \PY{o}{=} \PY{n}{np}\PY{o}{.}\PY{n}{sum}\PY{p}{(}\PY{n}{w} \PY{o}{*} \PY{n}{w}\PY{p}{,} \PY{n}{axis}\PY{o}{=}\PY{l+m+mi}{1}\PY{p}{)}
            \PY{n}{L} \PY{o}{=} \PY{n+nb}{len}\PY{p}{(}\PY{n}{we}\PY{p}{)}
            
            \PY{n}{ix} \PY{o}{=} \PY{l+m+mi}{0}
            \PY{k}{while} \PY{n}{ix} \PY{o}{\PYZlt{}} \PY{n}{L}\PY{p}{:}
                \PY{k}{while} \PY{n}{ix} \PY{o}{\PYZlt{}} \PY{n}{L} \PY{o+ow}{and} \PY{n}{we}\PY{p}{[}\PY{n}{ix}\PY{p}{]} \PY{o}{\PYZlt{}} \PY{n}{th}\PY{p}{:}
                    \PY{n}{ix} \PY{o}{=} \PY{n}{ix}\PY{o}{+}\PY{l+m+mi}{1}
                \PY{k}{if} \PY{n}{ix} \PY{o}{\PYZgt{}}\PY{o}{=} \PY{n}{L}\PY{p}{:}
                    \PY{k}{break}    \PY{c+c1}{\PYZsh{} ending on silence}
                \PY{n}{iy} \PY{o}{=} \PY{n}{ix}
                \PY{k}{while} \PY{n}{iy} \PY{o}{\PYZlt{}} \PY{n}{L} \PY{o+ow}{and} \PY{n}{we}\PY{p}{[}\PY{n}{iy}\PY{p}{]} \PY{o}{\PYZgt{}} \PY{n}{th}\PY{p}{:}
                    \PY{n}{iy} \PY{o}{=} \PY{n}{iy}\PY{o}{+}\PY{l+m+mi}{1}
                \PY{n}{edges}\PY{o}{.}\PY{n}{append}\PY{p}{(}\PY{p}{(}\PY{n}{ix} \PY{o}{*} \PY{n}{win}\PY{p}{,} \PY{n}{iy} \PY{o}{*} \PY{n}{win}\PY{p}{)}\PY{p}{)}
                \PY{n}{ix} \PY{o}{=} \PY{n}{iy}
            
            \PY{k}{return} \PY{n}{edges}
\end{Verbatim}


    \begin{Verbatim}[commandchars=\\\{\}]
{\color{incolor}In [{\color{incolor}7}]:} \PY{n+nb}{print} \PY{n}{dtmf\PYZus{}split}\PY{p}{(}\PY{n}{x}\PY{p}{)}
\end{Verbatim}


    \begin{Verbatim}[commandchars=\\\{\}]
[(0, 2400), (4800, 7200), (9600, 12000), (14400, 16800), (19200, 21600), (24000, 26400), (28800, 31200)]

    \end{Verbatim}

    Looks good. Now that we have a splitter, let's run a DFT over the tone
sections and find the DTMF frequencies that are closest to the peaks of
the DFT magnitude. The "low" DTMF frequencies are in the 697 Hz to 941
Hz range, while the high frequencies in the 1209 Hz to 1477 Hz range, so
we will look for a DFT peak in each of those intervals. For instance,
let's look at the first tone, and let's look at the peaks in the DFT:

    \begin{Verbatim}[commandchars=\\\{\}]
{\color{incolor}In [{\color{incolor}8}]:} \PY{n}{X} \PY{o}{=} \PY{n+nb}{abs}\PY{p}{(}\PY{n}{np}\PY{o}{.}\PY{n}{fft}\PY{o}{.}\PY{n}{fft}\PY{p}{(}\PY{n}{x}\PY{p}{[}\PY{l+m+mi}{0}\PY{p}{:}\PY{l+m+mi}{2400}\PY{p}{]}\PY{p}{)}\PY{p}{)}
        \PY{n}{plt}\PY{o}{.}\PY{n}{plot}\PY{p}{(}\PY{n}{X}\PY{p}{[}\PY{l+m+mi}{0}\PY{p}{:}\PY{l+m+mi}{500}\PY{p}{]}\PY{p}{)}\PY{p}{;}
\end{Verbatim}


    \begin{center}
    \adjustimage{max size={0.9\linewidth}{0.9\paperheight}}{output_13_0.png}
    \end{center}
    { \hspace*{\fill} \\}
    
    We clearly have identifiable peaks. The only thing we need to pay
attention to is making sure that we map real-world frequencies to the
DFT plot correctly (and vice versa).

    \begin{Verbatim}[commandchars=\\\{\}]
{\color{incolor}In [{\color{incolor}9}]:} \PY{k}{def} \PY{n+nf}{dtmf\PYZus{}decode}\PY{p}{(}\PY{n}{x}\PY{p}{,} \PY{n}{edges} \PY{o}{=} \PY{k+kc}{None}\PY{p}{)}\PY{p}{:}
            \PY{c+c1}{\PYZsh{} the DTMF frequencies}
            \PY{n}{LO\PYZus{}FREQS} \PY{o}{=} \PY{n}{np}\PY{o}{.}\PY{n}{array}\PY{p}{(}\PY{p}{[}\PY{l+m+mf}{697.0}\PY{p}{,} \PY{l+m+mf}{770.0}\PY{p}{,} \PY{l+m+mf}{852.0}\PY{p}{,} \PY{l+m+mf}{941.0}\PY{p}{]}\PY{p}{)}
            \PY{n}{HI\PYZus{}FREQS} \PY{o}{=} \PY{n}{np}\PY{o}{.}\PY{n}{array}\PY{p}{(}\PY{p}{[}\PY{l+m+mf}{1209.0}\PY{p}{,} \PY{l+m+mf}{1336.0}\PY{p}{,} \PY{l+m+mf}{1477.0}\PY{p}{]}\PY{p}{)}
        
            \PY{n}{KEYS} \PY{o}{=} \PY{p}{[}\PY{p}{[}\PY{l+s+s1}{\PYZsq{}}\PY{l+s+s1}{1}\PY{l+s+s1}{\PYZsq{}}\PY{p}{,} \PY{l+s+s1}{\PYZsq{}}\PY{l+s+s1}{2}\PY{l+s+s1}{\PYZsq{}}\PY{p}{,} \PY{l+s+s1}{\PYZsq{}}\PY{l+s+s1}{3}\PY{l+s+s1}{\PYZsq{}}\PY{p}{]}\PY{p}{,} \PY{p}{[}\PY{l+s+s1}{\PYZsq{}}\PY{l+s+s1}{4}\PY{l+s+s1}{\PYZsq{}}\PY{p}{,} \PY{l+s+s1}{\PYZsq{}}\PY{l+s+s1}{5}\PY{l+s+s1}{\PYZsq{}}\PY{p}{,} \PY{l+s+s1}{\PYZsq{}}\PY{l+s+s1}{6}\PY{l+s+s1}{\PYZsq{}}\PY{p}{]}\PY{p}{,} \PY{p}{[}\PY{l+s+s1}{\PYZsq{}}\PY{l+s+s1}{7}\PY{l+s+s1}{\PYZsq{}}\PY{p}{,} \PY{l+s+s1}{\PYZsq{}}\PY{l+s+s1}{8}\PY{l+s+s1}{\PYZsq{}}\PY{p}{,} \PY{l+s+s1}{\PYZsq{}}\PY{l+s+s1}{9}\PY{l+s+s1}{\PYZsq{}}\PY{p}{]}\PY{p}{,} \PY{p}{[}\PY{l+s+s1}{\PYZsq{}}\PY{l+s+s1}{*}\PY{l+s+s1}{\PYZsq{}}\PY{p}{,} \PY{l+s+s1}{\PYZsq{}}\PY{l+s+s1}{0}\PY{l+s+s1}{\PYZsq{}}\PY{p}{,} \PY{l+s+s1}{\PYZsq{}}\PY{l+s+s1}{\PYZsh{}}\PY{l+s+s1}{\PYZsq{}}\PY{p}{]}\PY{p}{]}
            
            \PY{c+c1}{\PYZsh{} frequency ranges to search for low and high DTMF tones}
            \PY{n}{LO\PYZus{}RANGE} \PY{o}{=} \PY{p}{(}\PY{l+m+mf}{680.0}\PY{p}{,} \PY{l+m+mf}{960.0}\PY{p}{)}
            \PY{n}{HI\PYZus{}RANGE} \PY{o}{=} \PY{p}{(}\PY{l+m+mf}{1180.0}\PY{p}{,} \PY{l+m+mf}{1500.0}\PY{p}{)}
        
            \PY{n}{number} \PY{o}{=} \PY{p}{[}\PY{p}{]}
            
            \PY{c+c1}{\PYZsh{} now examine each tone in turn. the freqency mapping on the DFT}
            \PY{c+c1}{\PYZsh{}  axis will be dependent on the length of the data vector}
            \PY{k}{if} \PY{n}{edges} \PY{o+ow}{is} \PY{k+kc}{None}\PY{p}{:}
                \PY{n}{edges} \PY{o}{=} \PY{n}{dtmf\PYZus{}split}\PY{p}{(}\PY{n}{x}\PY{p}{)}
            \PY{k}{for} \PY{n}{g} \PY{o+ow}{in} \PY{n}{edges}\PY{p}{:}
                \PY{c+c1}{\PYZsh{} compute the DFT of the tone segment}
                \PY{n}{X} \PY{o}{=} \PY{n+nb}{abs}\PY{p}{(}\PY{n}{np}\PY{o}{.}\PY{n}{fft}\PY{o}{.}\PY{n}{fft}\PY{p}{(}\PY{n}{x}\PY{p}{[}\PY{n}{g}\PY{p}{[}\PY{l+m+mi}{0}\PY{p}{]}\PY{p}{:}\PY{n}{g}\PY{p}{[}\PY{l+m+mi}{1}\PY{p}{]}\PY{p}{]}\PY{p}{)}\PY{p}{)}
                \PY{n}{N} \PY{o}{=} \PY{n+nb}{len}\PY{p}{(}\PY{n}{X}\PY{p}{)}
                \PY{c+c1}{\PYZsh{} compute the resolution in Hz of a DFT bin}
                \PY{n}{res} \PY{o}{=} \PY{n+nb}{float}\PY{p}{(}\PY{n}{FS}\PY{p}{)} \PY{o}{/} \PY{n}{N}
                
                \PY{c+c1}{\PYZsh{} find the peak location within the low freq range}
                \PY{n}{a} \PY{o}{=} \PY{n+nb}{int}\PY{p}{(}\PY{n}{LO\PYZus{}RANGE}\PY{p}{[}\PY{l+m+mi}{0}\PY{p}{]} \PY{o}{/} \PY{n}{res}\PY{p}{)}
                \PY{n}{b} \PY{o}{=} \PY{n+nb}{int}\PY{p}{(}\PY{n}{LO\PYZus{}RANGE}\PY{p}{[}\PY{l+m+mi}{1}\PY{p}{]} \PY{o}{/} \PY{n}{res}\PY{p}{)}
                \PY{n}{lo} \PY{o}{=} \PY{n}{a} \PY{o}{+} \PY{n}{np}\PY{o}{.}\PY{n}{argmax}\PY{p}{(}\PY{n}{X}\PY{p}{[}\PY{n}{a}\PY{p}{:}\PY{n}{b}\PY{p}{]}\PY{p}{)}
                \PY{c+c1}{\PYZsh{} find the peak location within the high freq range}
                \PY{n}{a} \PY{o}{=} \PY{n+nb}{int}\PY{p}{(}\PY{n}{HI\PYZus{}RANGE}\PY{p}{[}\PY{l+m+mi}{0}\PY{p}{]} \PY{o}{/} \PY{n}{res}\PY{p}{)}
                \PY{n}{b} \PY{o}{=} \PY{n+nb}{int}\PY{p}{(}\PY{n}{HI\PYZus{}RANGE}\PY{p}{[}\PY{l+m+mi}{1}\PY{p}{]} \PY{o}{/} \PY{n}{res}\PY{p}{)}
                \PY{n}{hi} \PY{o}{=} \PY{n}{a} \PY{o}{+} \PY{n}{np}\PY{o}{.}\PY{n}{argmax}\PY{p}{(}\PY{n}{X}\PY{p}{[}\PY{n}{a}\PY{p}{:}\PY{n}{b}\PY{p}{]}\PY{p}{)}
              
                \PY{c+c1}{\PYZsh{} now match the results to the DTMF frequencies}
                \PY{n}{row} \PY{o}{=} \PY{n}{np}\PY{o}{.}\PY{n}{argmin}\PY{p}{(}\PY{n+nb}{abs}\PY{p}{(}\PY{n}{LO\PYZus{}FREQS} \PY{o}{\PYZhy{}} \PY{n}{lo} \PY{o}{*} \PY{n}{res}\PY{p}{)}\PY{p}{)}
                \PY{n}{col} \PY{o}{=} \PY{n}{np}\PY{o}{.}\PY{n}{argmin}\PY{p}{(}\PY{n+nb}{abs}\PY{p}{(}\PY{n}{HI\PYZus{}FREQS} \PY{o}{\PYZhy{}} \PY{n}{hi} \PY{o}{*} \PY{n}{res}\PY{p}{)}\PY{p}{)}
        
                \PY{c+c1}{\PYZsh{} and finally convert that to the pressed key}
                \PY{n}{number}\PY{o}{.}\PY{n}{append}\PY{p}{(}\PY{n}{KEYS}\PY{p}{[}\PY{n}{row}\PY{p}{]}\PY{p}{[}\PY{n}{col}\PY{p}{]}\PY{p}{)}
            \PY{k}{return} \PY{n}{number}
\end{Verbatim}


    \begin{Verbatim}[commandchars=\\\{\}]
{\color{incolor}In [{\color{incolor}10}]:} \PY{n}{dtmf\PYZus{}decode}\PY{p}{(}\PY{n}{x}\PY{p}{)}
\end{Verbatim}


\begin{Verbatim}[commandchars=\\\{\}]
{\color{outcolor}Out[{\color{outcolor}10}]:} ['1', '2', '3', '\#', '\#', '4', '5']
\end{Verbatim}
            
    Yay! It works! As always, in communication systems, the receiver is much
more complicated than the receiver.

Of course this is a very simplified setup and we have glossed over a lot
of practical details. For instance, in the splitting function, the
thresholds are not determined dynamically and this may create problems
in the presence of noise. Similarly, we just detect a frequency peak in
the spectrum, but noise may make things more complicated.

For instance, listen to the following noise-corrupted version of the
original signal. Although the tones are still detectable by ear, the
segmentation algorithm fails and returns a single digit.

    \begin{Verbatim}[commandchars=\\\{\}]
{\color{incolor}In [{\color{incolor}11}]:} \PY{n}{noisy} \PY{o}{=} \PY{n}{x} \PY{o}{+} \PY{n}{np}\PY{o}{.}\PY{n}{random}\PY{o}{.}\PY{n}{uniform}\PY{p}{(}\PY{o}{\PYZhy{}}\PY{l+m+mi}{2}\PY{p}{,} \PY{l+m+mi}{2}\PY{p}{,} \PY{n+nb}{len}\PY{p}{(}\PY{n}{x}\PY{p}{)}\PY{p}{)}
         
         \PY{n}{IPython}\PY{o}{.}\PY{n}{display}\PY{o}{.}\PY{n}{Audio}\PY{p}{(}\PY{n}{noisy}\PY{p}{,} \PY{n}{rate}\PY{o}{=}\PY{n}{FS}\PY{p}{)}
\end{Verbatim}


\begin{Verbatim}[commandchars=\\\{\}]
{\color{outcolor}Out[{\color{outcolor}11}]:} <IPython.lib.display.Audio object>
\end{Verbatim}
            
    \begin{Verbatim}[commandchars=\\\{\}]
{\color{incolor}In [{\color{incolor}13}]:} \PY{n}{dtmf\PYZus{}decode}\PY{p}{(}\PY{n}{noisy}\PY{p}{)}
\end{Verbatim}


\begin{Verbatim}[commandchars=\\\{\}]
{\color{outcolor}Out[{\color{outcolor}13}]:} ['3']
\end{Verbatim}
            
    If we \textbf{carefully} change the segmentation threshold, we can still
decode

    \begin{Verbatim}[commandchars=\\\{\}]
{\color{incolor}In [{\color{incolor}14}]:} \PY{n}{dtmf\PYZus{}decode}\PY{p}{(}\PY{n}{x}\PY{p}{,} \PY{n}{dtmf\PYZus{}split}\PY{p}{(}\PY{n}{x}\PY{p}{,} \PY{n}{th}\PY{o}{=}\PY{l+m+mi}{220}\PY{p}{)}\PY{p}{)}
\end{Verbatim}


\begin{Verbatim}[commandchars=\\\{\}]
{\color{outcolor}Out[{\color{outcolor}14}]:} ['1', '2', '3', '\#', '\#', '4', '5']
\end{Verbatim}
            
    but if we're not careful...

    \begin{Verbatim}[commandchars=\\\{\}]
{\color{incolor}In [{\color{incolor}15}]:} \PY{n}{dtmf\PYZus{}decode}\PY{p}{(}\PY{n}{x}\PY{p}{,} \PY{n}{dtmf\PYZus{}split}\PY{p}{(}\PY{n}{x}\PY{p}{,} \PY{n}{th}\PY{o}{=}\PY{l+m+mi}{250}\PY{p}{)}\PY{p}{)}
\end{Verbatim}


\begin{Verbatim}[commandchars=\\\{\}]
{\color{outcolor}Out[{\color{outcolor}15}]:} ['2', '2', '5', '5', '5', '5', '5', '5', '4', '4', '5', '5']
\end{Verbatim}
            
    The sensitivity to the segmentation threshold confirms the fact that
segmentation should be performed using more sophisticated techniques,
which what happens in practical systems.

    \begin{Verbatim}[commandchars=\\\{\}]
{\color{incolor}In [{\color{incolor} }]:} 
\end{Verbatim}



    % Add a bibliography block to the postdoc
    
    
    
    \end{document}
