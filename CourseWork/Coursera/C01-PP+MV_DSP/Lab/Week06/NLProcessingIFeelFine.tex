
% Default to the notebook output style

    


% Inherit from the specified cell style.




    
\documentclass[11pt]{article}

    
    
    \usepackage[T1]{fontenc}
    % Nicer default font (+ math font) than Computer Modern for most use cases
    \usepackage{mathpazo}

    % Basic figure setup, for now with no caption control since it's done
    % automatically by Pandoc (which extracts ![](path) syntax from Markdown).
    \usepackage{graphicx}
    % We will generate all images so they have a width \maxwidth. This means
    % that they will get their normal width if they fit onto the page, but
    % are scaled down if they would overflow the margins.
    \makeatletter
    \def\maxwidth{\ifdim\Gin@nat@width>\linewidth\linewidth
    \else\Gin@nat@width\fi}
    \makeatother
    \let\Oldincludegraphics\includegraphics
    % Set max figure width to be 80% of text width, for now hardcoded.
    \renewcommand{\includegraphics}[1]{\Oldincludegraphics[width=.8\maxwidth]{#1}}
    % Ensure that by default, figures have no caption (until we provide a
    % proper Figure object with a Caption API and a way to capture that
    % in the conversion process - todo).
    \usepackage{caption}
    \DeclareCaptionLabelFormat{nolabel}{}
    \captionsetup{labelformat=nolabel}

    \usepackage{adjustbox} % Used to constrain images to a maximum size 
    \usepackage{xcolor} % Allow colors to be defined
    \usepackage{enumerate} % Needed for markdown enumerations to work
    \usepackage{geometry} % Used to adjust the document margins
    \usepackage{amsmath} % Equations
    \usepackage{amssymb} % Equations
    \usepackage{textcomp} % defines textquotesingle
    % Hack from http://tex.stackexchange.com/a/47451/13684:
    \AtBeginDocument{%
        \def\PYZsq{\textquotesingle}% Upright quotes in Pygmentized code
    }
    \usepackage{upquote} % Upright quotes for verbatim code
    \usepackage{eurosym} % defines \euro
    \usepackage[mathletters]{ucs} % Extended unicode (utf-8) support
    \usepackage[utf8x]{inputenc} % Allow utf-8 characters in the tex document
    \usepackage{fancyvrb} % verbatim replacement that allows latex
    \usepackage{grffile} % extends the file name processing of package graphics 
                         % to support a larger range 
    % The hyperref package gives us a pdf with properly built
    % internal navigation ('pdf bookmarks' for the table of contents,
    % internal cross-reference links, web links for URLs, etc.)
    \usepackage{hyperref}
    \usepackage{longtable} % longtable support required by pandoc >1.10
    \usepackage{booktabs}  % table support for pandoc > 1.12.2
    \usepackage[inline]{enumitem} % IRkernel/repr support (it uses the enumerate* environment)
    \usepackage[normalem]{ulem} % ulem is needed to support strikethroughs (\sout)
                                % normalem makes italics be italics, not underlines
    

    
    
    % Colors for the hyperref package
    \definecolor{urlcolor}{rgb}{0,.145,.698}
    \definecolor{linkcolor}{rgb}{.71,0.21,0.01}
    \definecolor{citecolor}{rgb}{.12,.54,.11}

    % ANSI colors
    \definecolor{ansi-black}{HTML}{3E424D}
    \definecolor{ansi-black-intense}{HTML}{282C36}
    \definecolor{ansi-red}{HTML}{E75C58}
    \definecolor{ansi-red-intense}{HTML}{B22B31}
    \definecolor{ansi-green}{HTML}{00A250}
    \definecolor{ansi-green-intense}{HTML}{007427}
    \definecolor{ansi-yellow}{HTML}{DDB62B}
    \definecolor{ansi-yellow-intense}{HTML}{B27D12}
    \definecolor{ansi-blue}{HTML}{208FFB}
    \definecolor{ansi-blue-intense}{HTML}{0065CA}
    \definecolor{ansi-magenta}{HTML}{D160C4}
    \definecolor{ansi-magenta-intense}{HTML}{A03196}
    \definecolor{ansi-cyan}{HTML}{60C6C8}
    \definecolor{ansi-cyan-intense}{HTML}{258F8F}
    \definecolor{ansi-white}{HTML}{C5C1B4}
    \definecolor{ansi-white-intense}{HTML}{A1A6B2}

    % commands and environments needed by pandoc snippets
    % extracted from the output of `pandoc -s`
    \providecommand{\tightlist}{%
      \setlength{\itemsep}{0pt}\setlength{\parskip}{0pt}}
    \DefineVerbatimEnvironment{Highlighting}{Verbatim}{commandchars=\\\{\}}
    % Add ',fontsize=\small' for more characters per line
    \newenvironment{Shaded}{}{}
    \newcommand{\KeywordTok}[1]{\textcolor[rgb]{0.00,0.44,0.13}{\textbf{{#1}}}}
    \newcommand{\DataTypeTok}[1]{\textcolor[rgb]{0.56,0.13,0.00}{{#1}}}
    \newcommand{\DecValTok}[1]{\textcolor[rgb]{0.25,0.63,0.44}{{#1}}}
    \newcommand{\BaseNTok}[1]{\textcolor[rgb]{0.25,0.63,0.44}{{#1}}}
    \newcommand{\FloatTok}[1]{\textcolor[rgb]{0.25,0.63,0.44}{{#1}}}
    \newcommand{\CharTok}[1]{\textcolor[rgb]{0.25,0.44,0.63}{{#1}}}
    \newcommand{\StringTok}[1]{\textcolor[rgb]{0.25,0.44,0.63}{{#1}}}
    \newcommand{\CommentTok}[1]{\textcolor[rgb]{0.38,0.63,0.69}{\textit{{#1}}}}
    \newcommand{\OtherTok}[1]{\textcolor[rgb]{0.00,0.44,0.13}{{#1}}}
    \newcommand{\AlertTok}[1]{\textcolor[rgb]{1.00,0.00,0.00}{\textbf{{#1}}}}
    \newcommand{\FunctionTok}[1]{\textcolor[rgb]{0.02,0.16,0.49}{{#1}}}
    \newcommand{\RegionMarkerTok}[1]{{#1}}
    \newcommand{\ErrorTok}[1]{\textcolor[rgb]{1.00,0.00,0.00}{\textbf{{#1}}}}
    \newcommand{\NormalTok}[1]{{#1}}
    
    % Additional commands for more recent versions of Pandoc
    \newcommand{\ConstantTok}[1]{\textcolor[rgb]{0.53,0.00,0.00}{{#1}}}
    \newcommand{\SpecialCharTok}[1]{\textcolor[rgb]{0.25,0.44,0.63}{{#1}}}
    \newcommand{\VerbatimStringTok}[1]{\textcolor[rgb]{0.25,0.44,0.63}{{#1}}}
    \newcommand{\SpecialStringTok}[1]{\textcolor[rgb]{0.73,0.40,0.53}{{#1}}}
    \newcommand{\ImportTok}[1]{{#1}}
    \newcommand{\DocumentationTok}[1]{\textcolor[rgb]{0.73,0.13,0.13}{\textit{{#1}}}}
    \newcommand{\AnnotationTok}[1]{\textcolor[rgb]{0.38,0.63,0.69}{\textbf{\textit{{#1}}}}}
    \newcommand{\CommentVarTok}[1]{\textcolor[rgb]{0.38,0.63,0.69}{\textbf{\textit{{#1}}}}}
    \newcommand{\VariableTok}[1]{\textcolor[rgb]{0.10,0.09,0.49}{{#1}}}
    \newcommand{\ControlFlowTok}[1]{\textcolor[rgb]{0.00,0.44,0.13}{\textbf{{#1}}}}
    \newcommand{\OperatorTok}[1]{\textcolor[rgb]{0.40,0.40,0.40}{{#1}}}
    \newcommand{\BuiltInTok}[1]{{#1}}
    \newcommand{\ExtensionTok}[1]{{#1}}
    \newcommand{\PreprocessorTok}[1]{\textcolor[rgb]{0.74,0.48,0.00}{{#1}}}
    \newcommand{\AttributeTok}[1]{\textcolor[rgb]{0.49,0.56,0.16}{{#1}}}
    \newcommand{\InformationTok}[1]{\textcolor[rgb]{0.38,0.63,0.69}{\textbf{\textit{{#1}}}}}
    \newcommand{\WarningTok}[1]{\textcolor[rgb]{0.38,0.63,0.69}{\textbf{\textit{{#1}}}}}
    
    
    % Define a nice break command that doesn't care if a line doesn't already
    % exist.
    \def\br{\hspace*{\fill} \\* }
    % Math Jax compatability definitions
    \def\gt{>}
    \def\lt{<}
    % Document parameters
    \title{FeelFine}
    
    
    

    % Pygments definitions
    
\makeatletter
\def\PY@reset{\let\PY@it=\relax \let\PY@bf=\relax%
    \let\PY@ul=\relax \let\PY@tc=\relax%
    \let\PY@bc=\relax \let\PY@ff=\relax}
\def\PY@tok#1{\csname PY@tok@#1\endcsname}
\def\PY@toks#1+{\ifx\relax#1\empty\else%
    \PY@tok{#1}\expandafter\PY@toks\fi}
\def\PY@do#1{\PY@bc{\PY@tc{\PY@ul{%
    \PY@it{\PY@bf{\PY@ff{#1}}}}}}}
\def\PY#1#2{\PY@reset\PY@toks#1+\relax+\PY@do{#2}}

\expandafter\def\csname PY@tok@w\endcsname{\def\PY@tc##1{\textcolor[rgb]{0.73,0.73,0.73}{##1}}}
\expandafter\def\csname PY@tok@c\endcsname{\let\PY@it=\textit\def\PY@tc##1{\textcolor[rgb]{0.25,0.50,0.50}{##1}}}
\expandafter\def\csname PY@tok@cp\endcsname{\def\PY@tc##1{\textcolor[rgb]{0.74,0.48,0.00}{##1}}}
\expandafter\def\csname PY@tok@k\endcsname{\let\PY@bf=\textbf\def\PY@tc##1{\textcolor[rgb]{0.00,0.50,0.00}{##1}}}
\expandafter\def\csname PY@tok@kp\endcsname{\def\PY@tc##1{\textcolor[rgb]{0.00,0.50,0.00}{##1}}}
\expandafter\def\csname PY@tok@kt\endcsname{\def\PY@tc##1{\textcolor[rgb]{0.69,0.00,0.25}{##1}}}
\expandafter\def\csname PY@tok@o\endcsname{\def\PY@tc##1{\textcolor[rgb]{0.40,0.40,0.40}{##1}}}
\expandafter\def\csname PY@tok@ow\endcsname{\let\PY@bf=\textbf\def\PY@tc##1{\textcolor[rgb]{0.67,0.13,1.00}{##1}}}
\expandafter\def\csname PY@tok@nb\endcsname{\def\PY@tc##1{\textcolor[rgb]{0.00,0.50,0.00}{##1}}}
\expandafter\def\csname PY@tok@nf\endcsname{\def\PY@tc##1{\textcolor[rgb]{0.00,0.00,1.00}{##1}}}
\expandafter\def\csname PY@tok@nc\endcsname{\let\PY@bf=\textbf\def\PY@tc##1{\textcolor[rgb]{0.00,0.00,1.00}{##1}}}
\expandafter\def\csname PY@tok@nn\endcsname{\let\PY@bf=\textbf\def\PY@tc##1{\textcolor[rgb]{0.00,0.00,1.00}{##1}}}
\expandafter\def\csname PY@tok@ne\endcsname{\let\PY@bf=\textbf\def\PY@tc##1{\textcolor[rgb]{0.82,0.25,0.23}{##1}}}
\expandafter\def\csname PY@tok@nv\endcsname{\def\PY@tc##1{\textcolor[rgb]{0.10,0.09,0.49}{##1}}}
\expandafter\def\csname PY@tok@no\endcsname{\def\PY@tc##1{\textcolor[rgb]{0.53,0.00,0.00}{##1}}}
\expandafter\def\csname PY@tok@nl\endcsname{\def\PY@tc##1{\textcolor[rgb]{0.63,0.63,0.00}{##1}}}
\expandafter\def\csname PY@tok@ni\endcsname{\let\PY@bf=\textbf\def\PY@tc##1{\textcolor[rgb]{0.60,0.60,0.60}{##1}}}
\expandafter\def\csname PY@tok@na\endcsname{\def\PY@tc##1{\textcolor[rgb]{0.49,0.56,0.16}{##1}}}
\expandafter\def\csname PY@tok@nt\endcsname{\let\PY@bf=\textbf\def\PY@tc##1{\textcolor[rgb]{0.00,0.50,0.00}{##1}}}
\expandafter\def\csname PY@tok@nd\endcsname{\def\PY@tc##1{\textcolor[rgb]{0.67,0.13,1.00}{##1}}}
\expandafter\def\csname PY@tok@s\endcsname{\def\PY@tc##1{\textcolor[rgb]{0.73,0.13,0.13}{##1}}}
\expandafter\def\csname PY@tok@sd\endcsname{\let\PY@it=\textit\def\PY@tc##1{\textcolor[rgb]{0.73,0.13,0.13}{##1}}}
\expandafter\def\csname PY@tok@si\endcsname{\let\PY@bf=\textbf\def\PY@tc##1{\textcolor[rgb]{0.73,0.40,0.53}{##1}}}
\expandafter\def\csname PY@tok@se\endcsname{\let\PY@bf=\textbf\def\PY@tc##1{\textcolor[rgb]{0.73,0.40,0.13}{##1}}}
\expandafter\def\csname PY@tok@sr\endcsname{\def\PY@tc##1{\textcolor[rgb]{0.73,0.40,0.53}{##1}}}
\expandafter\def\csname PY@tok@ss\endcsname{\def\PY@tc##1{\textcolor[rgb]{0.10,0.09,0.49}{##1}}}
\expandafter\def\csname PY@tok@sx\endcsname{\def\PY@tc##1{\textcolor[rgb]{0.00,0.50,0.00}{##1}}}
\expandafter\def\csname PY@tok@m\endcsname{\def\PY@tc##1{\textcolor[rgb]{0.40,0.40,0.40}{##1}}}
\expandafter\def\csname PY@tok@gh\endcsname{\let\PY@bf=\textbf\def\PY@tc##1{\textcolor[rgb]{0.00,0.00,0.50}{##1}}}
\expandafter\def\csname PY@tok@gu\endcsname{\let\PY@bf=\textbf\def\PY@tc##1{\textcolor[rgb]{0.50,0.00,0.50}{##1}}}
\expandafter\def\csname PY@tok@gd\endcsname{\def\PY@tc##1{\textcolor[rgb]{0.63,0.00,0.00}{##1}}}
\expandafter\def\csname PY@tok@gi\endcsname{\def\PY@tc##1{\textcolor[rgb]{0.00,0.63,0.00}{##1}}}
\expandafter\def\csname PY@tok@gr\endcsname{\def\PY@tc##1{\textcolor[rgb]{1.00,0.00,0.00}{##1}}}
\expandafter\def\csname PY@tok@ge\endcsname{\let\PY@it=\textit}
\expandafter\def\csname PY@tok@gs\endcsname{\let\PY@bf=\textbf}
\expandafter\def\csname PY@tok@gp\endcsname{\let\PY@bf=\textbf\def\PY@tc##1{\textcolor[rgb]{0.00,0.00,0.50}{##1}}}
\expandafter\def\csname PY@tok@go\endcsname{\def\PY@tc##1{\textcolor[rgb]{0.53,0.53,0.53}{##1}}}
\expandafter\def\csname PY@tok@gt\endcsname{\def\PY@tc##1{\textcolor[rgb]{0.00,0.27,0.87}{##1}}}
\expandafter\def\csname PY@tok@err\endcsname{\def\PY@bc##1{\setlength{\fboxsep}{0pt}\fcolorbox[rgb]{1.00,0.00,0.00}{1,1,1}{\strut ##1}}}
\expandafter\def\csname PY@tok@kc\endcsname{\let\PY@bf=\textbf\def\PY@tc##1{\textcolor[rgb]{0.00,0.50,0.00}{##1}}}
\expandafter\def\csname PY@tok@kd\endcsname{\let\PY@bf=\textbf\def\PY@tc##1{\textcolor[rgb]{0.00,0.50,0.00}{##1}}}
\expandafter\def\csname PY@tok@kn\endcsname{\let\PY@bf=\textbf\def\PY@tc##1{\textcolor[rgb]{0.00,0.50,0.00}{##1}}}
\expandafter\def\csname PY@tok@kr\endcsname{\let\PY@bf=\textbf\def\PY@tc##1{\textcolor[rgb]{0.00,0.50,0.00}{##1}}}
\expandafter\def\csname PY@tok@bp\endcsname{\def\PY@tc##1{\textcolor[rgb]{0.00,0.50,0.00}{##1}}}
\expandafter\def\csname PY@tok@fm\endcsname{\def\PY@tc##1{\textcolor[rgb]{0.00,0.00,1.00}{##1}}}
\expandafter\def\csname PY@tok@vc\endcsname{\def\PY@tc##1{\textcolor[rgb]{0.10,0.09,0.49}{##1}}}
\expandafter\def\csname PY@tok@vg\endcsname{\def\PY@tc##1{\textcolor[rgb]{0.10,0.09,0.49}{##1}}}
\expandafter\def\csname PY@tok@vi\endcsname{\def\PY@tc##1{\textcolor[rgb]{0.10,0.09,0.49}{##1}}}
\expandafter\def\csname PY@tok@vm\endcsname{\def\PY@tc##1{\textcolor[rgb]{0.10,0.09,0.49}{##1}}}
\expandafter\def\csname PY@tok@sa\endcsname{\def\PY@tc##1{\textcolor[rgb]{0.73,0.13,0.13}{##1}}}
\expandafter\def\csname PY@tok@sb\endcsname{\def\PY@tc##1{\textcolor[rgb]{0.73,0.13,0.13}{##1}}}
\expandafter\def\csname PY@tok@sc\endcsname{\def\PY@tc##1{\textcolor[rgb]{0.73,0.13,0.13}{##1}}}
\expandafter\def\csname PY@tok@dl\endcsname{\def\PY@tc##1{\textcolor[rgb]{0.73,0.13,0.13}{##1}}}
\expandafter\def\csname PY@tok@s2\endcsname{\def\PY@tc##1{\textcolor[rgb]{0.73,0.13,0.13}{##1}}}
\expandafter\def\csname PY@tok@sh\endcsname{\def\PY@tc##1{\textcolor[rgb]{0.73,0.13,0.13}{##1}}}
\expandafter\def\csname PY@tok@s1\endcsname{\def\PY@tc##1{\textcolor[rgb]{0.73,0.13,0.13}{##1}}}
\expandafter\def\csname PY@tok@mb\endcsname{\def\PY@tc##1{\textcolor[rgb]{0.40,0.40,0.40}{##1}}}
\expandafter\def\csname PY@tok@mf\endcsname{\def\PY@tc##1{\textcolor[rgb]{0.40,0.40,0.40}{##1}}}
\expandafter\def\csname PY@tok@mh\endcsname{\def\PY@tc##1{\textcolor[rgb]{0.40,0.40,0.40}{##1}}}
\expandafter\def\csname PY@tok@mi\endcsname{\def\PY@tc##1{\textcolor[rgb]{0.40,0.40,0.40}{##1}}}
\expandafter\def\csname PY@tok@il\endcsname{\def\PY@tc##1{\textcolor[rgb]{0.40,0.40,0.40}{##1}}}
\expandafter\def\csname PY@tok@mo\endcsname{\def\PY@tc##1{\textcolor[rgb]{0.40,0.40,0.40}{##1}}}
\expandafter\def\csname PY@tok@ch\endcsname{\let\PY@it=\textit\def\PY@tc##1{\textcolor[rgb]{0.25,0.50,0.50}{##1}}}
\expandafter\def\csname PY@tok@cm\endcsname{\let\PY@it=\textit\def\PY@tc##1{\textcolor[rgb]{0.25,0.50,0.50}{##1}}}
\expandafter\def\csname PY@tok@cpf\endcsname{\let\PY@it=\textit\def\PY@tc##1{\textcolor[rgb]{0.25,0.50,0.50}{##1}}}
\expandafter\def\csname PY@tok@c1\endcsname{\let\PY@it=\textit\def\PY@tc##1{\textcolor[rgb]{0.25,0.50,0.50}{##1}}}
\expandafter\def\csname PY@tok@cs\endcsname{\let\PY@it=\textit\def\PY@tc##1{\textcolor[rgb]{0.25,0.50,0.50}{##1}}}

\def\PYZbs{\char`\\}
\def\PYZus{\char`\_}
\def\PYZob{\char`\{}
\def\PYZcb{\char`\}}
\def\PYZca{\char`\^}
\def\PYZam{\char`\&}
\def\PYZlt{\char`\<}
\def\PYZgt{\char`\>}
\def\PYZsh{\char`\#}
\def\PYZpc{\char`\%}
\def\PYZdl{\char`\$}
\def\PYZhy{\char`\-}
\def\PYZsq{\char`\'}
\def\PYZdq{\char`\"}
\def\PYZti{\char`\~}
% for compatibility with earlier versions
\def\PYZat{@}
\def\PYZlb{[}
\def\PYZrb{]}
\makeatother


    % Exact colors from NB
    \definecolor{incolor}{rgb}{0.0, 0.0, 0.5}
    \definecolor{outcolor}{rgb}{0.545, 0.0, 0.0}



    
    % Prevent overflowing lines due to hard-to-break entities
    \sloppy 
    % Setup hyperref package
    \hypersetup{
      breaklinks=true,  % so long urls are correctly broken across lines
      colorlinks=true,
      urlcolor=urlcolor,
      linkcolor=linkcolor,
      citecolor=citecolor,
      }
    % Slightly bigger margins than the latex defaults
    
    \geometry{verbose,tmargin=1in,bmargin=1in,lmargin=1in,rmargin=1in}
    
    

    \begin{document}
    
    
    \maketitle
    
    

    
    \section{I Feel Fine, Digitally}\label{i-feel-fine-digitally}

    This notebook is a little musing inspired by the "internal mechanics" of
a very famous audio snippet, namely the guitar sound at the beginning of
the song "I Feel Fine", written and recorded by the Beatles in 1964. The
historical significance of the record lies in the following claim by
John Lennon:

\begin{quote}
\emph{"I defy anybody to find a record... unless it is some old blues
record from 1922... that uses feedback that way. So I claim it for the
Beatles. Before Hendrix, before The Who, before anybody. The first
feedback on record."}
\end{quote}

In this notebook, we are going to look in detail at this famous first
instance of recorded guitar feedback and we will try to set up a digital
model of what went down in the recording studio on that fateful 18
October 1964. In doing so we will look at a guitar simulator, at an amp
model and at the mechanics of feedback. But, before anything else, let's
listen to what this is all about:

    \begin{Verbatim}[commandchars=\\\{\}]
{\color{incolor}In [{\color{incolor}1}]:} \PY{o}{\PYZpc{}}\PY{k}{matplotlib} inline
        \PY{k+kn}{import} \PY{n+nn}{matplotlib}
        \PY{k+kn}{import} \PY{n+nn}{matplotlib}\PY{n+nn}{.}\PY{n+nn}{pyplot} \PY{k}{as} \PY{n+nn}{plt}
        \PY{k+kn}{import} \PY{n+nn}{numpy} \PY{k}{as} \PY{n+nn}{np}
        \PY{k+kn}{import} \PY{n+nn}{IPython}
        \PY{k+kn}{from} \PY{n+nn}{scipy}\PY{n+nn}{.}\PY{n+nn}{io} \PY{k}{import} \PY{n}{wavfile}
        \PY{k+kn}{from} \PY{n+nn}{IPython}\PY{n+nn}{.}\PY{n+nn}{display} \PY{k}{import} \PY{n}{Image}
\end{Verbatim}


    \begin{Verbatim}[commandchars=\\\{\}]
{\color{incolor}In [{\color{incolor}2}]:} \PY{n}{plt}\PY{o}{.}\PY{n}{rcParams}\PY{p}{[}\PY{l+s+s1}{\PYZsq{}}\PY{l+s+s1}{figure.figsize}\PY{l+s+s1}{\PYZsq{}}\PY{p}{]} \PY{o}{=} \PY{l+m+mi}{14}\PY{p}{,} \PY{l+m+mi}{4} 
\end{Verbatim}


    \begin{Verbatim}[commandchars=\\\{\}]
{\color{incolor}In [{\color{incolor}5}]:} \PY{n}{display}\PY{p}{(}\PY{n}{Image}\PY{p}{(}\PY{n}{filename}\PY{o}{=}\PY{l+s+s1}{\PYZsq{}}\PY{l+s+s1}{beatles.jpg}\PY{l+s+s1}{\PYZsq{}}\PY{p}{,} \PY{n}{width}\PY{o}{=}\PY{l+m+mi}{400}\PY{p}{)}\PY{p}{)}
        
        \PY{c+c1}{\PYZsh{} fs will be the global \PYZdq{}clock\PYZdq{} of all discrete\PYZhy{}time systems in this notebook}
        \PY{n}{fs}\PY{p}{,} \PY{n}{data} \PY{o}{=} \PY{n}{wavfile}\PY{o}{.}\PY{n}{read}\PY{p}{(}\PY{l+s+s2}{\PYZdq{}}\PY{l+s+s2}{iff.wav}\PY{l+s+s2}{\PYZdq{}}\PY{p}{)}
        \PY{c+c1}{\PYZsh{} bring the 16bit wav samples into the [\PYZhy{}1, 1] range}
        \PY{n}{data} \PY{o}{=} \PY{n}{data} \PY{o}{/} \PY{l+m+mf}{32767.0}
        \PY{n}{IPython}\PY{o}{.}\PY{n}{display}\PY{o}{.}\PY{n}{Audio}\PY{p}{(}\PY{n}{data}\PY{o}{=}\PY{n}{data}\PY{p}{,} \PY{n}{rate}\PY{o}{=}\PY{n}{fs}\PY{p}{,} \PY{n}{embed}\PY{o}{=}\PY{k+kc}{True}\PY{p}{)}
\end{Verbatim}


    \begin{center}
    \adjustimage{max size={0.9\linewidth}{0.9\paperheight}}{output_4_0.jpeg}
    \end{center}
    { \hspace*{\fill} \\}
    
\begin{Verbatim}[commandchars=\\\{\}]
{\color{outcolor}Out[{\color{outcolor}5}]:} <IPython.lib.display.Audio object>
\end{Verbatim}
            
    According to recording studio accounts, the sound was obtained by first
playing the A string on Lennon's Gibson semiacoustic guitar and then by
placing the guitar close to the amplifier. Indeed, the first two seconds
of the clip sound like a standard decaying guitar tone; after that the
feedback kicks in. The feedback, sometimes described as an "electric
razor" buzz, is caused by two phenomena: the sound generated by the
amplifier "hits" the A string and increases its vibration, and the
resulting increased signal drives the amplifier into saturation.

Schematically, these are the systems involved in the generation of the
opening of the song:

    \begin{Verbatim}[commandchars=\\\{\}]
{\color{incolor}In [{\color{incolor}7}]:} \PY{n}{display}\PY{p}{(}\PY{n}{Image}\PY{p}{(}\PY{n}{filename}\PY{o}{=}\PY{l+s+s1}{\PYZsq{}}\PY{l+s+s1}{bd.jpg}\PY{l+s+s1}{\PYZsq{}}\PY{p}{,} \PY{n}{width}\PY{o}{=}\PY{l+m+mi}{600}\PY{p}{)}\PY{p}{)}
\end{Verbatim}


    \begin{center}
    \adjustimage{max size={0.9\linewidth}{0.9\paperheight}}{output_6_0.jpeg}
    \end{center}
    { \hspace*{\fill} \\}
    
    In order to simulate this setup digitally, we need to come up with
resonable models for:

\begin{itemize}
\tightlist
\item
  the guitar \(G\), including the possibility of driving the string
  vibration during oscillation
\item
  the amplifier \(A\), including a saturating nonlinearity
\item
  the feedback channel \(F\), which will depend on the distance from the
  guitar to the amplifier
\end{itemize}

Let's examine each component in more detail.

    \subsection{1 - simulating a guitar}\label{simulating-a-guitar}

    Although we have already studied the Karplus-Strong algorithm as an
effective way to simulate a plucked sound, in this case we need a model
that is closer to the actual physics of a guitar, since we'll need to
drive the string oscillation in the feedback loop.

In a guitar, the sound is generated by the oscillation of strings that
are both under tension and fixed at both ends. Under these conditions, a
displacement of the string from its rest position (i.e. the initial
"plucking") will result in an oscillatory behavior in which the energy
imparted by the plucking travels back and forth between the ends of the
string in the form of standing waves. The natural modes of oscillation
of a string are all multiples of the string's fundamental frequency,
which is determined by its length, its mass and its tension (see, for
instance, \href{http://www.phys.unsw.edu.au/jw/strings.html}{here} for a
detailed explanation). This image (courtesy of
\href{http://en.wikipedia.org/wiki/Vibrating_string}{Wikipedia}) shows a
few oscillation modes on a string:

These vibrations are propagated to the body of an acoustic guitar and
converted into sound pressure waves or, for an electric guitar, they are
converted into an electrical waveform by the guitar's pickups.

We can appreciate this behavior in the initial (non feedback) portion of
the "I Feel Fine" sound snippet; let's first look at the waveform in the
time domain:

    \begin{Verbatim}[commandchars=\\\{\}]
{\color{incolor}In [{\color{incolor}8}]:} \PY{n}{plt}\PY{o}{.}\PY{n}{plot}\PY{p}{(}\PY{n}{data}\PY{p}{)}\PY{p}{;}
        \PY{n}{plt}\PY{o}{.}\PY{n}{xlabel}\PY{p}{(}\PY{l+s+s2}{\PYZdq{}}\PY{l+s+s2}{sample}\PY{l+s+s2}{\PYZdq{}}\PY{p}{)}\PY{p}{;}
        \PY{n}{plt}\PY{o}{.}\PY{n}{ylabel}\PY{p}{(}\PY{l+s+s2}{\PYZdq{}}\PY{l+s+s2}{amplitude}\PY{l+s+s2}{\PYZdq{}}\PY{p}{)}\PY{p}{;}
\end{Verbatim}


    \begin{center}
    \adjustimage{max size={0.9\linewidth}{0.9\paperheight}}{output_10_0.png}
    \end{center}
    { \hspace*{\fill} \\}
    
    The "pure guitar" part is approximately from sample 10000 to sample
40000. If we plot the spectrum of this portion:

    \begin{Verbatim}[commandchars=\\\{\}]
{\color{incolor}In [{\color{incolor}9}]:} \PY{n}{s} \PY{o}{=} \PY{n+nb}{abs}\PY{p}{(}\PY{n}{np}\PY{o}{.}\PY{n}{fft}\PY{o}{.}\PY{n}{fftpack}\PY{o}{.}\PY{n}{fft}\PY{p}{(}\PY{n}{data}\PY{p}{[}\PY{l+m+mi}{10000}\PY{p}{:}\PY{l+m+mi}{40000}\PY{p}{]}\PY{p}{)}\PY{p}{)}\PY{p}{;}
        \PY{n}{s} \PY{o}{=} \PY{n}{s}\PY{p}{[}\PY{l+m+mi}{0}\PY{p}{:}\PY{n+nb}{int}\PY{p}{(}\PY{n+nb}{len}\PY{p}{(}\PY{n}{s}\PY{p}{)}\PY{o}{/}\PY{l+m+mi}{2}\PY{p}{)}\PY{p}{]}
        \PY{n}{plt}\PY{o}{.}\PY{n}{plot}\PY{p}{(}\PY{n}{np}\PY{o}{.}\PY{n}{linspace}\PY{p}{(}\PY{l+m+mi}{0}\PY{p}{,}\PY{l+m+mi}{1}\PY{p}{,}\PY{n+nb}{len}\PY{p}{(}\PY{n}{s}\PY{p}{)}\PY{p}{)}\PY{o}{*}\PY{p}{(}\PY{n}{fs}\PY{o}{/}\PY{l+m+mi}{2}\PY{p}{)}\PY{p}{,} \PY{n}{s}\PY{p}{)}\PY{p}{;}
        \PY{n}{plt}\PY{o}{.}\PY{n}{xlabel}\PY{p}{(}\PY{l+s+s2}{\PYZdq{}}\PY{l+s+s2}{frequency (Hz)}\PY{l+s+s2}{\PYZdq{}}\PY{p}{)}\PY{p}{;}
        \PY{n}{plt}\PY{o}{.}\PY{n}{ylabel}\PY{p}{(}\PY{l+s+s2}{\PYZdq{}}\PY{l+s+s2}{magnitude}\PY{l+s+s2}{\PYZdq{}}\PY{p}{)}\PY{p}{;}
\end{Verbatim}


    \begin{center}
    \adjustimage{max size={0.9\linewidth}{0.9\paperheight}}{output_12_0.png}
    \end{center}
    { \hspace*{\fill} \\}
    
    Indeed we can see that the frequency content of the sound contains
multiples of a fundamental frequency at 110Hz, which corresponds to the
open A string on a standard-tuning guitar.

From a signal processing point of view, the guitar string acts as a
resonator resonating at several multiples of a fundamental frequency;
this fundamental frequency determines the \emph{pitch} of the played
note. In the digital domain, we know we can implement a resonator at a
single frequency \(\omega_0\) with a second-order IIR of the form

\[
  H(z) = \frac{1}{(1 - \rho e^{j\omega_0}z^{-1})(1 - \rho e^{-j\omega_0}z^{-1})}, \quad \rho \approx 1
\]

i.e. by placing a pair of complex-conjugate poles close to the unit
circle at an angle \(\pm\omega_0\). A simple extension of this concept,
which places poles at \emph{all} multiples of a fundamental frequency,
is the \textbf{comb filter}. A comb filter of order \(N\) has the
transfer function

\[
  H(z) = \frac{1 - \rho z^{-1}}{1 - \rho^N z^{-N}}
\]

It is easy to see that the poles of the filters are at
\(z_k = \rho e^{j\frac{2\pi}{N}k}\), except for \(k=0\) where the zero
cancels the pole. For example, here is the frequency response of
\(H(z) = 1/(1 - (0.99)^N z^{-N})\) for \(N=9\):

    \begin{Verbatim}[commandchars=\\\{\}]
{\color{incolor}In [{\color{incolor}10}]:} \PY{k+kn}{from} \PY{n+nn}{scipy} \PY{k}{import} \PY{n}{signal}
         
         \PY{n}{w}\PY{p}{,} \PY{n}{h} \PY{o}{=} \PY{n}{signal}\PY{o}{.}\PY{n}{freqz}\PY{p}{(}\PY{l+m+mi}{1}\PY{p}{,} \PY{p}{[}\PY{l+m+mi}{1}\PY{p}{,} \PY{l+m+mi}{0}\PY{p}{,} \PY{l+m+mi}{0}\PY{p}{,} \PY{l+m+mi}{0}\PY{p}{,} \PY{l+m+mi}{0}\PY{p}{,} \PY{l+m+mi}{0}\PY{p}{,} \PY{l+m+mi}{0}\PY{p}{,} \PY{l+m+mi}{0}\PY{p}{,} \PY{l+m+mi}{0}\PY{p}{,} \PY{o}{\PYZhy{}}\PY{o}{.}\PY{l+m+mi}{99}\PY{o}{*}\PY{o}{*}\PY{l+m+mi}{9}\PY{p}{]}\PY{p}{)}
         \PY{n}{plt}\PY{o}{.}\PY{n}{plot}\PY{p}{(}\PY{n}{w}\PY{p}{,} \PY{n+nb}{abs}\PY{p}{(}\PY{n}{h}\PY{p}{)}\PY{p}{)}\PY{p}{;}
\end{Verbatim}


    \begin{center}
    \adjustimage{max size={0.9\linewidth}{0.9\paperheight}}{output_14_0.png}
    \end{center}
    { \hspace*{\fill} \\}
    
    An added advantage of the comb filter is that it is very easy to
implement, since it requires only two multiplication per output sample
\emph{independently} of \(N\):

\[
    y[n] = \rho^N y[n-N] + x[n] - \rho x[n-1]
\]

With this, here's an idea for a guitar simulation: the string behavior
is captured by a comb filter where \(N\) is given by the period (in
samples) of the desired fundamental frequency. Let's try it out:

    \begin{Verbatim}[commandchars=\\\{\}]
{\color{incolor}In [{\color{incolor}11}]:} \PY{k}{class} \PY{n+nc}{guitar}\PY{p}{:}
             \PY{k}{def} \PY{n+nf}{\PYZus{}\PYZus{}init\PYZus{}\PYZus{}}\PY{p}{(}\PY{n+nb+bp}{self}\PY{p}{,} \PY{n}{pitch}\PY{o}{=}\PY{l+m+mi}{110}\PY{p}{,} \PY{n}{fs}\PY{o}{=}\PY{l+m+mi}{24000}\PY{p}{)}\PY{p}{:}
                 \PY{c+c1}{\PYZsh{} init the class with desired pitch and underlying sampling frequency}
                 \PY{n+nb+bp}{self}\PY{o}{.}\PY{n}{M} \PY{o}{=} \PY{n+nb}{int}\PY{p}{(}\PY{n}{np}\PY{o}{.}\PY{n}{round}\PY{p}{(}\PY{n}{fs} \PY{o}{/} \PY{n}{pitch}\PY{p}{)}\PY{p}{)} \PY{c+c1}{\PYZsh{} fundamental period in samples}
                 \PY{n+nb+bp}{self}\PY{o}{.}\PY{n}{R} \PY{o}{=} \PY{l+m+mf}{0.9999}               \PY{c+c1}{\PYZsh{} decay factor}
                 \PY{n+nb+bp}{self}\PY{o}{.}\PY{n}{RM} \PY{o}{=} \PY{n+nb+bp}{self}\PY{o}{.}\PY{n}{R} \PY{o}{*}\PY{o}{*} \PY{n+nb+bp}{self}\PY{o}{.}\PY{n}{M} 
                 \PY{n+nb+bp}{self}\PY{o}{.}\PY{n}{ybuf} \PY{o}{=} \PY{n}{np}\PY{o}{.}\PY{n}{zeros}\PY{p}{(}\PY{n+nb+bp}{self}\PY{o}{.}\PY{n}{M}\PY{p}{)}  \PY{c+c1}{\PYZsh{} output buffer (circular)}
                 \PY{n+nb+bp}{self}\PY{o}{.}\PY{n}{iy} \PY{o}{=} \PY{l+m+mi}{0}                   \PY{c+c1}{\PYZsh{} index into out buf}
                 \PY{n+nb+bp}{self}\PY{o}{.}\PY{n}{xbuf} \PY{o}{=} \PY{l+m+mi}{0}                 \PY{c+c1}{\PYZsh{} input buffer (just one sample)}
                 
             \PY{k}{def} \PY{n+nf}{play}\PY{p}{(}\PY{n+nb+bp}{self}\PY{p}{,} \PY{n}{x}\PY{p}{)}\PY{p}{:}
                 \PY{n}{y} \PY{o}{=} \PY{n}{np}\PY{o}{.}\PY{n}{zeros}\PY{p}{(}\PY{n+nb}{len}\PY{p}{(}\PY{n}{x}\PY{p}{)}\PY{p}{)}
                 \PY{k}{for} \PY{n}{n} \PY{o+ow}{in} \PY{n+nb}{range}\PY{p}{(}\PY{n+nb}{len}\PY{p}{(}\PY{n}{x}\PY{p}{)}\PY{p}{)}\PY{p}{:}
                     \PY{n}{t} \PY{o}{=} \PY{n}{x}\PY{p}{[}\PY{n}{n}\PY{p}{]} \PY{o}{\PYZhy{}} \PY{n+nb+bp}{self}\PY{o}{.}\PY{n}{R} \PY{o}{*} \PY{n+nb+bp}{self}\PY{o}{.}\PY{n}{xbuf} \PY{o}{+} \PY{n+nb+bp}{self}\PY{o}{.}\PY{n}{RM} \PY{o}{*} \PY{n+nb+bp}{self}\PY{o}{.}\PY{n}{ybuf}\PY{p}{[}\PY{n+nb+bp}{self}\PY{o}{.}\PY{n}{iy}\PY{p}{]}
                     \PY{n+nb+bp}{self}\PY{o}{.}\PY{n}{ybuf}\PY{p}{[}\PY{n+nb+bp}{self}\PY{o}{.}\PY{n}{iy}\PY{p}{]} \PY{o}{=} \PY{n}{t}
                     \PY{n+nb+bp}{self}\PY{o}{.}\PY{n}{iy} \PY{o}{=} \PY{p}{(}\PY{n+nb+bp}{self}\PY{o}{.}\PY{n}{iy} \PY{o}{+} \PY{l+m+mi}{1}\PY{p}{)} \PY{o}{\PYZpc{}} \PY{n+nb+bp}{self}\PY{o}{.}\PY{n}{M}
                     \PY{n+nb+bp}{self}\PY{o}{.}\PY{n}{xbuf} \PY{o}{=} \PY{n}{x}\PY{p}{[}\PY{n}{n}\PY{p}{]}
                     \PY{n}{y}\PY{p}{[}\PY{n}{n}\PY{p}{]} \PY{o}{=} \PY{n}{t}
                 \PY{k}{return} \PY{n}{y}
\end{Verbatim}


    Now we model the string plucking as a simple impulse signal in zero and
we input that to the guitar model:

    \begin{Verbatim}[commandchars=\\\{\}]
{\color{incolor}In [{\color{incolor}12}]:} \PY{c+c1}{\PYZsh{} create a 2\PYZhy{}second signal}
         \PY{n}{d} \PY{o}{=} \PY{n}{np}\PY{o}{.}\PY{n}{zeros}\PY{p}{(}\PY{n}{fs}\PY{o}{*}\PY{l+m+mi}{2}\PY{p}{)}
         \PY{c+c1}{\PYZsh{} impulse in zero (string plucked)}
         \PY{n}{d}\PY{p}{[}\PY{l+m+mi}{0}\PY{p}{]} \PY{o}{=} \PY{l+m+mi}{1}
         
         \PY{c+c1}{\PYZsh{} create the A string}
         \PY{n}{y} \PY{o}{=} \PY{n}{guitar}\PY{p}{(}\PY{l+m+mi}{110}\PY{p}{,} \PY{n}{fs}\PY{p}{)}\PY{o}{.}\PY{n}{play}\PY{p}{(}\PY{n}{d}\PY{p}{)}
         \PY{n}{IPython}\PY{o}{.}\PY{n}{display}\PY{o}{.}\PY{n}{Audio}\PY{p}{(}\PY{n}{data}\PY{o}{=}\PY{n}{y}\PY{p}{,} \PY{n}{rate}\PY{o}{=}\PY{n}{fs}\PY{p}{,} \PY{n}{embed}\PY{o}{=}\PY{k+kc}{True}\PY{p}{)}
\end{Verbatim}


\begin{Verbatim}[commandchars=\\\{\}]
{\color{outcolor}Out[{\color{outcolor}12}]:} <IPython.lib.display.Audio object>
\end{Verbatim}
            
    Ouch! The pitch may be right but the timbre is grotesque! The reason
becomes self-evident if we look at the frequency content:

    \begin{Verbatim}[commandchars=\\\{\}]
{\color{incolor}In [{\color{incolor}13}]:} \PY{n}{s} \PY{o}{=} \PY{n+nb}{abs}\PY{p}{(}\PY{n}{np}\PY{o}{.}\PY{n}{fft}\PY{o}{.}\PY{n}{fftpack}\PY{o}{.}\PY{n}{fft}\PY{p}{(}\PY{n}{y}\PY{p}{)}\PY{p}{)}\PY{p}{;}
         \PY{n}{s} \PY{o}{=} \PY{n}{s}\PY{p}{[}\PY{l+m+mi}{0}\PY{p}{:}\PY{n+nb}{int}\PY{p}{(}\PY{n+nb}{len}\PY{p}{(}\PY{n}{s}\PY{p}{)}\PY{o}{/}\PY{l+m+mi}{2}\PY{p}{)}\PY{p}{]}
         \PY{n}{plt}\PY{o}{.}\PY{n}{plot}\PY{p}{(}\PY{n}{np}\PY{o}{.}\PY{n}{linspace}\PY{p}{(}\PY{l+m+mi}{0}\PY{p}{,}\PY{l+m+mi}{1}\PY{p}{,}\PY{n+nb}{len}\PY{p}{(}\PY{n}{s}\PY{p}{)}\PY{p}{)}\PY{o}{*}\PY{p}{(}\PY{n}{fs}\PY{o}{/}\PY{l+m+mi}{2}\PY{p}{)}\PY{p}{,} \PY{n}{s}\PY{p}{)}\PY{p}{;}
\end{Verbatim}


    \begin{center}
    \adjustimage{max size={0.9\linewidth}{0.9\paperheight}}{output_20_0.png}
    \end{center}
    { \hspace*{\fill} \\}
    
    Although we have multiples of the fundamental, we actually have
\emph{too many} spectral lines and, because of the zero in the filter, a
highpass characteristic. In a real-world guitar both the stiffness of
the string and the response of the guitar's body would limit the number
of harmonics to just a few, as we saw in the figure above where we
analyzed the snippet from the song.

Well, it's not too hard to get rid of unwanted spectral content: just
add a lowpass filter. In this case we use a simple Butterworth that
keeps only the first five harmonics:

    \begin{Verbatim}[commandchars=\\\{\}]
{\color{incolor}In [{\color{incolor}14}]:} \PY{k+kn}{from} \PY{n+nn}{scipy} \PY{k}{import} \PY{n}{signal}
         
         \PY{k}{class} \PY{n+nc}{guitar}\PY{p}{:}
             \PY{k}{def} \PY{n+nf}{\PYZus{}\PYZus{}init\PYZus{}\PYZus{}}\PY{p}{(}\PY{n+nb+bp}{self}\PY{p}{,} \PY{n}{pitch}\PY{o}{=}\PY{l+m+mi}{110}\PY{p}{,} \PY{n}{fs}\PY{o}{=}\PY{l+m+mi}{24000}\PY{p}{)}\PY{p}{:}
                 \PY{c+c1}{\PYZsh{} init the class with desired pitch and underlying sampling frequency}
                 \PY{n+nb+bp}{self}\PY{o}{.}\PY{n}{M} \PY{o}{=} \PY{n+nb}{int}\PY{p}{(}\PY{n}{np}\PY{o}{.}\PY{n}{round}\PY{p}{(}\PY{n}{fs} \PY{o}{/} \PY{n}{pitch}\PY{p}{)} \PY{p}{)}\PY{c+c1}{\PYZsh{} fundamental period in samples}
                 \PY{n+nb+bp}{self}\PY{o}{.}\PY{n}{R} \PY{o}{=} \PY{l+m+mf}{0.9999}               \PY{c+c1}{\PYZsh{} decay factor}
                 \PY{n+nb+bp}{self}\PY{o}{.}\PY{n}{RM} \PY{o}{=} \PY{n+nb+bp}{self}\PY{o}{.}\PY{n}{R} \PY{o}{*}\PY{o}{*} \PY{n+nb+bp}{self}\PY{o}{.}\PY{n}{M} 
                 \PY{n+nb+bp}{self}\PY{o}{.}\PY{n}{ybuf} \PY{o}{=} \PY{n}{np}\PY{o}{.}\PY{n}{zeros}\PY{p}{(}\PY{n+nb+bp}{self}\PY{o}{.}\PY{n}{M}\PY{p}{)}  \PY{c+c1}{\PYZsh{} output buffer (circular)}
                 \PY{n+nb+bp}{self}\PY{o}{.}\PY{n}{iy} \PY{o}{=} \PY{l+m+mi}{0}                   \PY{c+c1}{\PYZsh{} index into out buf}
                 \PY{n+nb+bp}{self}\PY{o}{.}\PY{n}{xbuf} \PY{o}{=} \PY{l+m+mi}{0}                 \PY{c+c1}{\PYZsh{} input buffer (just one sample)}
                 \PY{c+c1}{\PYZsh{} 6th\PYZhy{}order Butterworth, keep 5 harmonics:}
                 \PY{n+nb+bp}{self}\PY{o}{.}\PY{n}{bfb}\PY{p}{,} \PY{n+nb+bp}{self}\PY{o}{.}\PY{n}{bfa} \PY{o}{=} \PY{n}{signal}\PY{o}{.}\PY{n}{butter}\PY{p}{(}\PY{l+m+mi}{6}\PY{p}{,} \PY{n+nb}{min}\PY{p}{(}\PY{l+m+mf}{0.5}\PY{p}{,} \PY{l+m+mf}{5.0} \PY{o}{*} \PY{n}{pitch} \PY{o}{/} \PY{n}{fs}\PY{p}{)}\PY{p}{)}
                 \PY{n+nb+bp}{self}\PY{o}{.}\PY{n}{bfb} \PY{o}{*}\PY{o}{=} \PY{l+m+mi}{1000}              \PY{c+c1}{\PYZsh{} set a little gain }
                 \PY{c+c1}{\PYZsh{} initial conditions for the filter. We need this because we need to}
                 \PY{c+c1}{\PYZsh{} filter on a sample\PYZhy{}by\PYZhy{}sample basis later on}
                 \PY{n+nb+bp}{self}\PY{o}{.}\PY{n}{bfs} \PY{o}{=} \PY{n}{signal}\PY{o}{.}\PY{n}{lfiltic}\PY{p}{(}\PY{n+nb+bp}{self}\PY{o}{.}\PY{n}{bfb}\PY{p}{,} \PY{n+nb+bp}{self}\PY{o}{.}\PY{n}{bfa}\PY{p}{,} \PY{p}{[}\PY{l+m+mi}{0}\PY{p}{]}\PY{p}{)}
                 
             \PY{k}{def} \PY{n+nf}{play}\PY{p}{(}\PY{n+nb+bp}{self}\PY{p}{,} \PY{n}{x}\PY{p}{)}\PY{p}{:}
                 \PY{n}{y} \PY{o}{=} \PY{n}{np}\PY{o}{.}\PY{n}{zeros}\PY{p}{(}\PY{n+nb}{len}\PY{p}{(}\PY{n}{x}\PY{p}{)}\PY{p}{)}
                 \PY{k}{for} \PY{n}{n} \PY{o+ow}{in} \PY{n+nb}{range}\PY{p}{(}\PY{n+nb}{len}\PY{p}{(}\PY{n}{x}\PY{p}{)}\PY{p}{)}\PY{p}{:}
                     \PY{c+c1}{\PYZsh{} comb filter}
                     \PY{n}{t} \PY{o}{=} \PY{n}{x}\PY{p}{[}\PY{n}{n}\PY{p}{]} \PY{o}{\PYZhy{}} \PY{n+nb+bp}{self}\PY{o}{.}\PY{n}{R} \PY{o}{*} \PY{n+nb+bp}{self}\PY{o}{.}\PY{n}{xbuf} \PY{o}{+} \PY{n+nb+bp}{self}\PY{o}{.}\PY{n}{RM} \PY{o}{*} \PY{n+nb+bp}{self}\PY{o}{.}\PY{n}{ybuf}\PY{p}{[}\PY{n+nb+bp}{self}\PY{o}{.}\PY{n}{iy}\PY{p}{]}
                     \PY{n+nb+bp}{self}\PY{o}{.}\PY{n}{ybuf}\PY{p}{[}\PY{n+nb+bp}{self}\PY{o}{.}\PY{n}{iy}\PY{p}{]} \PY{o}{=} \PY{n}{t}
                     \PY{n+nb+bp}{self}\PY{o}{.}\PY{n}{iy} \PY{o}{=} \PY{p}{(}\PY{n+nb+bp}{self}\PY{o}{.}\PY{n}{iy} \PY{o}{+} \PY{l+m+mi}{1}\PY{p}{)} \PY{o}{\PYZpc{}} \PY{n+nb+bp}{self}\PY{o}{.}\PY{n}{M}
                     \PY{n+nb+bp}{self}\PY{o}{.}\PY{n}{xbuf} \PY{o}{=} \PY{n}{x}\PY{p}{[}\PY{n}{n}\PY{p}{]}
                     \PY{c+c1}{\PYZsh{} lowpass filter, keep filter status for next sample}
                     \PY{n}{y}\PY{p}{[}\PY{n}{n}\PY{p}{]}\PY{p}{,} \PY{n+nb+bp}{self}\PY{o}{.}\PY{n}{bfs} \PY{o}{=} \PY{n}{signal}\PY{o}{.}\PY{n}{lfilter}\PY{p}{(}\PY{n+nb+bp}{self}\PY{o}{.}\PY{n}{bfb}\PY{p}{,} \PY{n+nb+bp}{self}\PY{o}{.}\PY{n}{bfa}\PY{p}{,} \PY{p}{[}\PY{n}{t}\PY{p}{]}\PY{p}{,} \PY{n}{zi}\PY{o}{=}\PY{n+nb+bp}{self}\PY{o}{.}\PY{n}{bfs}\PY{p}{)}
                 \PY{k}{return} \PY{n}{y}
\end{Verbatim}


    OK, let's give it a spin:

    \begin{Verbatim}[commandchars=\\\{\}]
{\color{incolor}In [{\color{incolor}15}]:} \PY{n}{y} \PY{o}{=} \PY{n}{guitar}\PY{p}{(}\PY{l+m+mi}{110}\PY{p}{,} \PY{n}{fs}\PY{p}{)}\PY{o}{.}\PY{n}{play}\PY{p}{(}\PY{n}{d}\PY{p}{)}
         \PY{n}{IPython}\PY{o}{.}\PY{n}{display}\PY{o}{.}\PY{n}{Audio}\PY{p}{(}\PY{n}{data}\PY{o}{=}\PY{n}{y}\PY{p}{,} \PY{n}{rate}\PY{o}{=}\PY{n}{fs}\PY{p}{,} \PY{n}{embed}\PY{o}{=}\PY{k+kc}{True}\PY{p}{)}
\end{Verbatim}


\begin{Verbatim}[commandchars=\\\{\}]
{\color{outcolor}Out[{\color{outcolor}15}]:} <IPython.lib.display.Audio object>
\end{Verbatim}
            
    Ah, so much better, no? Almost like the real thing. We can check the
spectrum and indeed we're close to what we wanted; the guitar is in the
bag.

    \begin{Verbatim}[commandchars=\\\{\}]
{\color{incolor}In [{\color{incolor}16}]:} \PY{n}{s} \PY{o}{=} \PY{n+nb}{abs}\PY{p}{(}\PY{n}{np}\PY{o}{.}\PY{n}{fft}\PY{o}{.}\PY{n}{fftpack}\PY{o}{.}\PY{n}{fft}\PY{p}{(}\PY{n}{y}\PY{p}{[}\PY{l+m+mi}{10000}\PY{p}{:}\PY{l+m+mi}{30000}\PY{p}{]}\PY{p}{)}\PY{p}{)}\PY{p}{;}
         \PY{n}{s} \PY{o}{=} \PY{n}{s}\PY{p}{[}\PY{l+m+mi}{0}\PY{p}{:}\PY{n+nb}{int}\PY{p}{(}\PY{n+nb}{len}\PY{p}{(}\PY{n}{s}\PY{p}{)}\PY{o}{/}\PY{l+m+mi}{2}\PY{p}{)}\PY{p}{]}
         \PY{n}{plt}\PY{o}{.}\PY{n}{plot}\PY{p}{(}\PY{n}{np}\PY{o}{.}\PY{n}{linspace}\PY{p}{(}\PY{l+m+mi}{0}\PY{p}{,}\PY{l+m+mi}{1}\PY{p}{,}\PY{n+nb}{len}\PY{p}{(}\PY{n}{s}\PY{p}{)}\PY{p}{)}\PY{o}{*}\PY{p}{(}\PY{n}{fs}\PY{o}{/}\PY{l+m+mi}{2}\PY{p}{)}\PY{p}{,} \PY{n}{s}\PY{p}{)}\PY{p}{;}
\end{Verbatim}


    \begin{center}
    \adjustimage{max size={0.9\linewidth}{0.9\paperheight}}{output_26_0.png}
    \end{center}
    { \hspace*{\fill} \\}
    
    \subsection{2 - the amplifier}\label{the-amplifier}

    In the "I Feel Fine" setup, the volume of the amplifier remains
constant; however, because of the feedback, the input will keep
increasing and, at one point or another, any real-world amplifier will
be driven into saturation. When that happens, the output is no longer a
scaled version of the input but gets "clipped" to the maximum output
level allowed by the amp. We can easily simulate this behavior with a
simple memoryless clipping operator:

    \begin{Verbatim}[commandchars=\\\{\}]
{\color{incolor}In [{\color{incolor}17}]:} \PY{k}{def} \PY{n+nf}{amplify}\PY{p}{(}\PY{n}{x}\PY{p}{)}\PY{p}{:}
             \PY{n}{TH} \PY{o}{=} \PY{l+m+mf}{0.9}           \PY{c+c1}{\PYZsh{} threshold}
             \PY{n}{y} \PY{o}{=} \PY{n}{np}\PY{o}{.}\PY{n}{copy}\PY{p}{(}\PY{n}{x}\PY{p}{)}
             \PY{n}{y}\PY{p}{[}\PY{n}{y} \PY{o}{\PYZgt{}}  \PY{n}{TH}\PY{p}{]} \PY{o}{=}  \PY{n}{TH}
             \PY{n}{y}\PY{p}{[}\PY{n}{y} \PY{o}{\PYZlt{}} \PY{o}{\PYZhy{}}\PY{n}{TH}\PY{p}{]} \PY{o}{=} \PY{o}{\PYZhy{}}\PY{n}{TH}
             \PY{k}{return} \PY{n}{y}
\end{Verbatim}


    We can easily check the characteristic of the amplifier simulator:

    \begin{Verbatim}[commandchars=\\\{\}]
{\color{incolor}In [{\color{incolor}18}]:} \PY{n}{x} \PY{o}{=} \PY{n}{np}\PY{o}{.}\PY{n}{linspace}\PY{p}{(}\PY{o}{\PYZhy{}}\PY{l+m+mi}{2}\PY{p}{,} \PY{l+m+mi}{2}\PY{p}{,} \PY{l+m+mi}{100}\PY{p}{)}
         \PY{n}{plt}\PY{o}{.}\PY{n}{plot}\PY{p}{(}\PY{n}{x}\PY{p}{,} \PY{n}{amplify}\PY{p}{(}\PY{n}{x}\PY{p}{)}\PY{p}{)}\PY{p}{;}
         \PY{n}{plt}\PY{o}{.}\PY{n}{xlabel}\PY{p}{(}\PY{l+s+s2}{\PYZdq{}}\PY{l+s+s2}{input}\PY{l+s+s2}{\PYZdq{}}\PY{p}{)}\PY{p}{;}
         \PY{n}{plt}\PY{o}{.}\PY{n}{ylabel}\PY{p}{(}\PY{l+s+s2}{\PYZdq{}}\PY{l+s+s2}{output}\PY{l+s+s2}{\PYZdq{}}\PY{p}{)}\PY{p}{;}
\end{Verbatim}


    \begin{center}
    \adjustimage{max size={0.9\linewidth}{0.9\paperheight}}{output_31_0.png}
    \end{center}
    { \hspace*{\fill} \\}
    
    While the response is linear between -TH and TH, it is important to
remark that the clipping introduces a nonlinearity in the processing
chain. In the case of linear systems, sinusoids are eigenfunctions and
therefore a linear system can only alter a sinusoid by modifying its
amplitude and phase. This is not the case with nonlinear systems, which
can profoundly alter the spectrum of a signal by creating new
frequencies. While these effects are very difficult to analyze
mathematically, from the acoustic point of view nonlinear distortion can
be very interesting, and "I Feel Fine" is just one example amongst
countless others.

It is instructive at this point to look at the spectrogram (i.e. the
STFT) of the sound sample (figure obtained with a commercial audio
spectrum analyzer); note how, indeed, the spectral content shows many
more spectral lines after the nonlinearity of the amplifier comes into
play.

    \begin{Verbatim}[commandchars=\\\{\}]
{\color{incolor}In [{\color{incolor}21}]:} \PY{n}{display}\PY{p}{(}\PY{n}{Image}\PY{p}{(}\PY{n}{filename}\PY{o}{=}\PY{l+s+s1}{\PYZsq{}}\PY{l+s+s1}{specgram.png}\PY{l+s+s1}{\PYZsq{}}\PY{p}{,} \PY{n}{width}\PY{o}{=}\PY{l+m+mi}{800}\PY{p}{)}\PY{p}{)}
\end{Verbatim}


    \begin{center}
    \adjustimage{max size={0.9\linewidth}{0.9\paperheight}}{output_33_0.png}
    \end{center}
    { \hspace*{\fill} \\}
    
    \subsection{3 - the acoustic feedback}\label{the-acoustic-feedback}

    The last piece of the processing chain is the acoustic channel that
closes the feedback loop. The sound pressure waves generated by the
loudspeaker of the amplifier travel through the air and eventually reach
the vibrating string. For feedback to kick in, two things must happen:

\begin{itemize}
\tightlist
\item
  the energy transfer from the pressure wave to the vibrating string
  should be non-negligible
\item
  the phase of the vibrating string must be sufficiently aligned with
  the phase of the sound wave in order for the sound wave to "feed" the
  vibration.
\end{itemize}

Sound travels in the air at about 340 meters per second and sound
pressure decays with the reciprocal of the traveled distance. We can
build an elementary acoustic channel simulation by neglecting everything
except delay and attenuation. The output of the acoustic channel for a
guitar-amplifier distance of \(d\) meters will be therefore

\[
    y[n] = \alpha x[n-M]
\]

where \(\alpha = 1/d\) and \(M\) is the propagation delay in samples;
with an internal clock of \(F_s\) Hz we have
\(M = \lfloor d/(c F_s) \rfloor\) where \(c\) is the speed of sound.

    \begin{Verbatim}[commandchars=\\\{\}]
{\color{incolor}In [{\color{incolor}22}]:} \PY{k}{class} \PY{n+nc}{feedback}\PY{p}{:}
             \PY{n}{SPEED\PYZus{}OF\PYZus{}SOUND} \PY{o}{=} \PY{l+m+mf}{343.0} \PY{c+c1}{\PYZsh{} m/s}
             \PY{k}{def} \PY{n+nf}{\PYZus{}\PYZus{}init\PYZus{}\PYZus{}}\PY{p}{(}\PY{n+nb+bp}{self}\PY{p}{,} \PY{n}{max\PYZus{}distance\PYZus{}m} \PY{o}{=} \PY{l+m+mi}{5}\PY{p}{,} \PY{n}{fs}\PY{o}{=}\PY{l+m+mi}{24000}\PY{p}{)}\PY{p}{:}  
                 \PY{c+c1}{\PYZsh{} init class with maximum distance}
                 \PY{n+nb+bp}{self}\PY{o}{.}\PY{n}{L} \PY{o}{=} \PY{n+nb}{int}\PY{p}{(}\PY{n}{np}\PY{o}{.}\PY{n}{ceil}\PY{p}{(}\PY{n}{max\PYZus{}distance\PYZus{}m} \PY{o}{/} \PY{n+nb+bp}{self}\PY{o}{.}\PY{n}{SPEED\PYZus{}OF\PYZus{}SOUND} \PY{o}{*} \PY{n}{fs}\PY{p}{)}\PY{p}{)}\PY{p}{;}
                 \PY{n+nb+bp}{self}\PY{o}{.}\PY{n}{xbuf} \PY{o}{=} \PY{n}{np}\PY{o}{.}\PY{n}{zeros}\PY{p}{(}\PY{n+nb+bp}{self}\PY{o}{.}\PY{n}{L}\PY{p}{)}     \PY{c+c1}{\PYZsh{} circular buffer}
                 \PY{n+nb+bp}{self}\PY{o}{.}\PY{n}{ix} \PY{o}{=} \PY{l+m+mi}{0}
                 
             \PY{k}{def} \PY{n+nf}{get}\PY{p}{(}\PY{n+nb+bp}{self}\PY{p}{,} \PY{n}{x}\PY{p}{,} \PY{n}{distance}\PY{p}{)}\PY{p}{:}
                 \PY{n}{d} \PY{o}{=} \PY{n+nb}{int}\PY{p}{(}\PY{n}{np}\PY{o}{.}\PY{n}{ceil}\PY{p}{(}\PY{n}{distance} \PY{o}{/} \PY{n+nb+bp}{self}\PY{o}{.}\PY{n}{SPEED\PYZus{}OF\PYZus{}SOUND} \PY{o}{*} \PY{n}{fs}\PY{p}{)}\PY{p}{)}    \PY{c+c1}{\PYZsh{} delay in samples}
                 \PY{n+nb+bp}{self}\PY{o}{.}\PY{n}{xbuf}\PY{p}{[}\PY{n+nb+bp}{self}\PY{o}{.}\PY{n}{ix}\PY{p}{]} \PY{o}{=} \PY{n}{x}
                 \PY{n}{x} \PY{o}{=} \PY{n+nb+bp}{self}\PY{o}{.}\PY{n}{xbuf}\PY{p}{[}\PY{p}{(}\PY{n+nb+bp}{self}\PY{o}{.}\PY{n}{L} \PY{o}{+} \PY{n+nb+bp}{self}\PY{o}{.}\PY{n}{ix} \PY{o}{\PYZhy{}} \PY{n}{d}\PY{p}{)} \PY{o}{\PYZpc{}} \PY{n+nb+bp}{self}\PY{o}{.}\PY{n}{L}\PY{p}{]}
                 \PY{n+nb+bp}{self}\PY{o}{.}\PY{n}{ix} \PY{o}{=} \PY{p}{(}\PY{n+nb+bp}{self}\PY{o}{.}\PY{n}{ix} \PY{o}{+} \PY{l+m+mi}{1}\PY{p}{)} \PY{o}{\PYZpc{}} \PY{n+nb+bp}{self}\PY{o}{.}\PY{n}{L}
                 \PY{k}{return} \PY{n}{x} \PY{o}{/} \PY{n+nb}{float}\PY{p}{(}\PY{n}{distance}\PY{p}{)}
\end{Verbatim}


    \subsection{4 - play it, Johnny}\label{play-it-johnny}

    OK, we're ready to play. We will generate a few seconds of sound, one
sample at a time, following these steps:

\begin{itemize}
\tightlist
\item
  generate a guitar sample
\item
  process it with the nonlinear amplifier
\item
  feed it back to the guitar via the acoustic channel using a
  time-varying distance
\end{itemize}

During the simulation, we will change the distance used in the feedback
channel model to account for the fact that the guitar is first played at
a distance from the amplifier, and then it is placed very close to it.
In the first phase, the sound will simply be a decaying note and then
the feedback will start moving the string back in full swing and drive
the amp into saturation. We also need to introduce some coupling loss
between the sound pressure waves emitted by the loudspeaker and the
string, since air and wound steel have rather different impedences.

Let's see if that works:

    \begin{Verbatim}[commandchars=\\\{\}]
{\color{incolor}In [{\color{incolor}23}]:} \PY{n}{g} \PY{o}{=} \PY{n}{guitar}\PY{p}{(}\PY{l+m+mi}{110}\PY{p}{)}    \PY{c+c1}{\PYZsh{} the A string}
         \PY{n}{f} \PY{o}{=} \PY{n}{feedback}\PY{p}{(}\PY{p}{)}     \PY{c+c1}{\PYZsh{} the feedback channel}
         
         \PY{c+c1}{\PYZsh{} the \PYZdq{}coupling loss\PYZdq{} between air and string is high. Let\PYZsq{}s say that}
         \PY{c+c1}{\PYZsh{}  it is about 80dBs}
         \PY{n}{COUPLING\PYZus{}LOSS} \PY{o}{=} \PY{l+m+mf}{0.0001}
         
         \PY{c+c1}{\PYZsh{} John starts 3m away and then places the guitar basically against the amp}
         \PY{c+c1}{\PYZsh{}  after 1.5 seconds}
         \PY{n}{START\PYZus{}DISTANCE} \PY{o}{=} \PY{l+m+mi}{3} 
         \PY{n}{END\PYZus{}DISTANCE} \PY{o}{=} \PY{l+m+mf}{0.05}
         
         \PY{n}{N} \PY{o}{=} \PY{n+nb}{int}\PY{p}{(}\PY{n}{fs} \PY{o}{*} \PY{l+m+mi}{5}\PY{p}{)}         \PY{c+c1}{\PYZsh{} play for 5 seconds}
         \PY{n}{y} \PY{o}{=} \PY{n}{np}\PY{o}{.}\PY{n}{zeros}\PY{p}{(}\PY{n}{N}\PY{p}{)}     
         \PY{n}{x} \PY{o}{=} \PY{p}{[}\PY{l+m+mi}{1}\PY{p}{]}            \PY{c+c1}{\PYZsh{} the initial plucking}
         \PY{c+c1}{\PYZsh{} now we create each sample in a loop by processing the guitar sound}
         \PY{c+c1}{\PYZsh{} thru the amp and then feeding back the attenuated and delayed sound}
         \PY{c+c1}{\PYZsh{} to the guitar}
         \PY{k}{for} \PY{n}{n} \PY{o+ow}{in} \PY{n+nb}{range}\PY{p}{(}\PY{n}{N}\PY{p}{)}\PY{p}{:}
             \PY{n}{y}\PY{p}{[}\PY{n}{n}\PY{p}{]} \PY{o}{=} \PY{n}{amplify}\PY{p}{(}\PY{n}{g}\PY{o}{.}\PY{n}{play}\PY{p}{(}\PY{n}{x}\PY{p}{)}\PY{p}{)}
             \PY{n}{x} \PY{o}{=} \PY{p}{[}\PY{n}{COUPLING\PYZus{}LOSS} \PY{o}{*} \PY{n}{f}\PY{o}{.}\PY{n}{get}\PY{p}{(}\PY{n}{y}\PY{p}{[}\PY{n}{n}\PY{p}{]}\PY{p}{,} \PY{n}{START\PYZus{}DISTANCE} \PY{k}{if} \PY{n}{n} \PY{o}{\PYZlt{}} \PY{p}{(}\PY{l+m+mf}{1.5} \PY{o}{*} \PY{n}{fs}\PY{p}{)} \PY{k}{else} \PY{n}{END\PYZus{}DISTANCE}\PY{p}{)}\PY{p}{]}
                
                 
         \PY{n}{IPython}\PY{o}{.}\PY{n}{display}\PY{o}{.}\PY{n}{Audio}\PY{p}{(}\PY{n}{data}\PY{o}{=}\PY{n}{y}\PY{p}{,} \PY{n}{rate}\PY{o}{=}\PY{n}{fs}\PY{p}{,} \PY{n}{embed}\PY{o}{=}\PY{k+kc}{True}\PY{p}{)}
\end{Verbatim}


\begin{Verbatim}[commandchars=\\\{\}]
{\color{outcolor}Out[{\color{outcolor}23}]:} <IPython.lib.display.Audio object>
\end{Verbatim}
            
    Pretty close, no? Of course the sound is not as rich as the original
recording since

\begin{itemize}
\tightlist
\item
  real guitars and real amplifiers are very complex physical system with
  many more types of nonlinearities; amongst others:
\item
  the spectral content generated by the string varies with the amplitude
  of its oscillation
\item
  the spectrum of the generated sound is not perfectly harmonic due to
  the physical size of the string
\item
  the string may start touching the frets when driven into large
  oscillations
\item
  the loudspeaker may introduce additional frequencies if driven too
  hard
\item
  ...
\item
  we have neglected the full frequency response of the amp both in
  linear and in nonlinear mode
\item
  it's the BEATLES, man! How can DSP compete?
\end{itemize}

Well, hope this was a fun and instructive foray into music and signal
processing. You can now play with the parameters of the simulation and
try to find alternative setups:

\begin{itemize}
\tightlist
\item
  try to change the characteristic of the amp, maybe using a sigmoid
  (hyperbolic tangent)
\item
  change the gain, the coupling loss or the frequency of the guitar
\item
  change John's guitar's position and verify that feedback does not
  occur at all distances.
\end{itemize}

    \begin{Verbatim}[commandchars=\\\{\}]
{\color{incolor}In [{\color{incolor} }]:} 
\end{Verbatim}



    % Add a bibliography block to the postdoc
    
    
    
    \end{document}
