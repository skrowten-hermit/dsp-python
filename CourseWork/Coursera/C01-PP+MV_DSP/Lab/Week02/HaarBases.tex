
% Default to the notebook output style

    


% Inherit from the specified cell style.




    
\documentclass[11pt]{article}

    
    
    \usepackage[T1]{fontenc}
    % Nicer default font (+ math font) than Computer Modern for most use cases
    \usepackage{mathpazo}

    % Basic figure setup, for now with no caption control since it's done
    % automatically by Pandoc (which extracts ![](path) syntax from Markdown).
    \usepackage{graphicx}
    % We will generate all images so they have a width \maxwidth. This means
    % that they will get their normal width if they fit onto the page, but
    % are scaled down if they would overflow the margins.
    \makeatletter
    \def\maxwidth{\ifdim\Gin@nat@width>\linewidth\linewidth
    \else\Gin@nat@width\fi}
    \makeatother
    \let\Oldincludegraphics\includegraphics
    % Set max figure width to be 80% of text width, for now hardcoded.
    \renewcommand{\includegraphics}[1]{\Oldincludegraphics[width=.8\maxwidth]{#1}}
    % Ensure that by default, figures have no caption (until we provide a
    % proper Figure object with a Caption API and a way to capture that
    % in the conversion process - todo).
    \usepackage{caption}
    \DeclareCaptionLabelFormat{nolabel}{}
    \captionsetup{labelformat=nolabel}

    \usepackage{adjustbox} % Used to constrain images to a maximum size 
    \usepackage{xcolor} % Allow colors to be defined
    \usepackage{enumerate} % Needed for markdown enumerations to work
    \usepackage{geometry} % Used to adjust the document margins
    \usepackage{amsmath} % Equations
    \usepackage{amssymb} % Equations
    \usepackage{textcomp} % defines textquotesingle
    % Hack from http://tex.stackexchange.com/a/47451/13684:
    \AtBeginDocument{%
        \def\PYZsq{\textquotesingle}% Upright quotes in Pygmentized code
    }
    \usepackage{upquote} % Upright quotes for verbatim code
    \usepackage{eurosym} % defines \euro
    \usepackage[mathletters]{ucs} % Extended unicode (utf-8) support
    \usepackage[utf8x]{inputenc} % Allow utf-8 characters in the tex document
    \usepackage{fancyvrb} % verbatim replacement that allows latex
    \usepackage{grffile} % extends the file name processing of package graphics 
                         % to support a larger range 
    % The hyperref package gives us a pdf with properly built
    % internal navigation ('pdf bookmarks' for the table of contents,
    % internal cross-reference links, web links for URLs, etc.)
    \usepackage{hyperref}
    \usepackage{longtable} % longtable support required by pandoc >1.10
    \usepackage{booktabs}  % table support for pandoc > 1.12.2
    \usepackage[inline]{enumitem} % IRkernel/repr support (it uses the enumerate* environment)
    \usepackage[normalem]{ulem} % ulem is needed to support strikethroughs (\sout)
                                % normalem makes italics be italics, not underlines
    

    
    
    % Colors for the hyperref package
    \definecolor{urlcolor}{rgb}{0,.145,.698}
    \definecolor{linkcolor}{rgb}{.71,0.21,0.01}
    \definecolor{citecolor}{rgb}{.12,.54,.11}

    % ANSI colors
    \definecolor{ansi-black}{HTML}{3E424D}
    \definecolor{ansi-black-intense}{HTML}{282C36}
    \definecolor{ansi-red}{HTML}{E75C58}
    \definecolor{ansi-red-intense}{HTML}{B22B31}
    \definecolor{ansi-green}{HTML}{00A250}
    \definecolor{ansi-green-intense}{HTML}{007427}
    \definecolor{ansi-yellow}{HTML}{DDB62B}
    \definecolor{ansi-yellow-intense}{HTML}{B27D12}
    \definecolor{ansi-blue}{HTML}{208FFB}
    \definecolor{ansi-blue-intense}{HTML}{0065CA}
    \definecolor{ansi-magenta}{HTML}{D160C4}
    \definecolor{ansi-magenta-intense}{HTML}{A03196}
    \definecolor{ansi-cyan}{HTML}{60C6C8}
    \definecolor{ansi-cyan-intense}{HTML}{258F8F}
    \definecolor{ansi-white}{HTML}{C5C1B4}
    \definecolor{ansi-white-intense}{HTML}{A1A6B2}

    % commands and environments needed by pandoc snippets
    % extracted from the output of `pandoc -s`
    \providecommand{\tightlist}{%
      \setlength{\itemsep}{0pt}\setlength{\parskip}{0pt}}
    \DefineVerbatimEnvironment{Highlighting}{Verbatim}{commandchars=\\\{\}}
    % Add ',fontsize=\small' for more characters per line
    \newenvironment{Shaded}{}{}
    \newcommand{\KeywordTok}[1]{\textcolor[rgb]{0.00,0.44,0.13}{\textbf{{#1}}}}
    \newcommand{\DataTypeTok}[1]{\textcolor[rgb]{0.56,0.13,0.00}{{#1}}}
    \newcommand{\DecValTok}[1]{\textcolor[rgb]{0.25,0.63,0.44}{{#1}}}
    \newcommand{\BaseNTok}[1]{\textcolor[rgb]{0.25,0.63,0.44}{{#1}}}
    \newcommand{\FloatTok}[1]{\textcolor[rgb]{0.25,0.63,0.44}{{#1}}}
    \newcommand{\CharTok}[1]{\textcolor[rgb]{0.25,0.44,0.63}{{#1}}}
    \newcommand{\StringTok}[1]{\textcolor[rgb]{0.25,0.44,0.63}{{#1}}}
    \newcommand{\CommentTok}[1]{\textcolor[rgb]{0.38,0.63,0.69}{\textit{{#1}}}}
    \newcommand{\OtherTok}[1]{\textcolor[rgb]{0.00,0.44,0.13}{{#1}}}
    \newcommand{\AlertTok}[1]{\textcolor[rgb]{1.00,0.00,0.00}{\textbf{{#1}}}}
    \newcommand{\FunctionTok}[1]{\textcolor[rgb]{0.02,0.16,0.49}{{#1}}}
    \newcommand{\RegionMarkerTok}[1]{{#1}}
    \newcommand{\ErrorTok}[1]{\textcolor[rgb]{1.00,0.00,0.00}{\textbf{{#1}}}}
    \newcommand{\NormalTok}[1]{{#1}}
    
    % Additional commands for more recent versions of Pandoc
    \newcommand{\ConstantTok}[1]{\textcolor[rgb]{0.53,0.00,0.00}{{#1}}}
    \newcommand{\SpecialCharTok}[1]{\textcolor[rgb]{0.25,0.44,0.63}{{#1}}}
    \newcommand{\VerbatimStringTok}[1]{\textcolor[rgb]{0.25,0.44,0.63}{{#1}}}
    \newcommand{\SpecialStringTok}[1]{\textcolor[rgb]{0.73,0.40,0.53}{{#1}}}
    \newcommand{\ImportTok}[1]{{#1}}
    \newcommand{\DocumentationTok}[1]{\textcolor[rgb]{0.73,0.13,0.13}{\textit{{#1}}}}
    \newcommand{\AnnotationTok}[1]{\textcolor[rgb]{0.38,0.63,0.69}{\textbf{\textit{{#1}}}}}
    \newcommand{\CommentVarTok}[1]{\textcolor[rgb]{0.38,0.63,0.69}{\textbf{\textit{{#1}}}}}
    \newcommand{\VariableTok}[1]{\textcolor[rgb]{0.10,0.09,0.49}{{#1}}}
    \newcommand{\ControlFlowTok}[1]{\textcolor[rgb]{0.00,0.44,0.13}{\textbf{{#1}}}}
    \newcommand{\OperatorTok}[1]{\textcolor[rgb]{0.40,0.40,0.40}{{#1}}}
    \newcommand{\BuiltInTok}[1]{{#1}}
    \newcommand{\ExtensionTok}[1]{{#1}}
    \newcommand{\PreprocessorTok}[1]{\textcolor[rgb]{0.74,0.48,0.00}{{#1}}}
    \newcommand{\AttributeTok}[1]{\textcolor[rgb]{0.49,0.56,0.16}{{#1}}}
    \newcommand{\InformationTok}[1]{\textcolor[rgb]{0.38,0.63,0.69}{\textbf{\textit{{#1}}}}}
    \newcommand{\WarningTok}[1]{\textcolor[rgb]{0.38,0.63,0.69}{\textbf{\textit{{#1}}}}}
    
    
    % Define a nice break command that doesn't care if a line doesn't already
    % exist.
    \def\br{\hspace*{\fill} \\* }
    % Math Jax compatability definitions
    \def\gt{>}
    \def\lt{<}
    % Document parameters
    \title{hb}
    
    
    

    % Pygments definitions
    
\makeatletter
\def\PY@reset{\let\PY@it=\relax \let\PY@bf=\relax%
    \let\PY@ul=\relax \let\PY@tc=\relax%
    \let\PY@bc=\relax \let\PY@ff=\relax}
\def\PY@tok#1{\csname PY@tok@#1\endcsname}
\def\PY@toks#1+{\ifx\relax#1\empty\else%
    \PY@tok{#1}\expandafter\PY@toks\fi}
\def\PY@do#1{\PY@bc{\PY@tc{\PY@ul{%
    \PY@it{\PY@bf{\PY@ff{#1}}}}}}}
\def\PY#1#2{\PY@reset\PY@toks#1+\relax+\PY@do{#2}}

\expandafter\def\csname PY@tok@w\endcsname{\def\PY@tc##1{\textcolor[rgb]{0.73,0.73,0.73}{##1}}}
\expandafter\def\csname PY@tok@c\endcsname{\let\PY@it=\textit\def\PY@tc##1{\textcolor[rgb]{0.25,0.50,0.50}{##1}}}
\expandafter\def\csname PY@tok@cp\endcsname{\def\PY@tc##1{\textcolor[rgb]{0.74,0.48,0.00}{##1}}}
\expandafter\def\csname PY@tok@k\endcsname{\let\PY@bf=\textbf\def\PY@tc##1{\textcolor[rgb]{0.00,0.50,0.00}{##1}}}
\expandafter\def\csname PY@tok@kp\endcsname{\def\PY@tc##1{\textcolor[rgb]{0.00,0.50,0.00}{##1}}}
\expandafter\def\csname PY@tok@kt\endcsname{\def\PY@tc##1{\textcolor[rgb]{0.69,0.00,0.25}{##1}}}
\expandafter\def\csname PY@tok@o\endcsname{\def\PY@tc##1{\textcolor[rgb]{0.40,0.40,0.40}{##1}}}
\expandafter\def\csname PY@tok@ow\endcsname{\let\PY@bf=\textbf\def\PY@tc##1{\textcolor[rgb]{0.67,0.13,1.00}{##1}}}
\expandafter\def\csname PY@tok@nb\endcsname{\def\PY@tc##1{\textcolor[rgb]{0.00,0.50,0.00}{##1}}}
\expandafter\def\csname PY@tok@nf\endcsname{\def\PY@tc##1{\textcolor[rgb]{0.00,0.00,1.00}{##1}}}
\expandafter\def\csname PY@tok@nc\endcsname{\let\PY@bf=\textbf\def\PY@tc##1{\textcolor[rgb]{0.00,0.00,1.00}{##1}}}
\expandafter\def\csname PY@tok@nn\endcsname{\let\PY@bf=\textbf\def\PY@tc##1{\textcolor[rgb]{0.00,0.00,1.00}{##1}}}
\expandafter\def\csname PY@tok@ne\endcsname{\let\PY@bf=\textbf\def\PY@tc##1{\textcolor[rgb]{0.82,0.25,0.23}{##1}}}
\expandafter\def\csname PY@tok@nv\endcsname{\def\PY@tc##1{\textcolor[rgb]{0.10,0.09,0.49}{##1}}}
\expandafter\def\csname PY@tok@no\endcsname{\def\PY@tc##1{\textcolor[rgb]{0.53,0.00,0.00}{##1}}}
\expandafter\def\csname PY@tok@nl\endcsname{\def\PY@tc##1{\textcolor[rgb]{0.63,0.63,0.00}{##1}}}
\expandafter\def\csname PY@tok@ni\endcsname{\let\PY@bf=\textbf\def\PY@tc##1{\textcolor[rgb]{0.60,0.60,0.60}{##1}}}
\expandafter\def\csname PY@tok@na\endcsname{\def\PY@tc##1{\textcolor[rgb]{0.49,0.56,0.16}{##1}}}
\expandafter\def\csname PY@tok@nt\endcsname{\let\PY@bf=\textbf\def\PY@tc##1{\textcolor[rgb]{0.00,0.50,0.00}{##1}}}
\expandafter\def\csname PY@tok@nd\endcsname{\def\PY@tc##1{\textcolor[rgb]{0.67,0.13,1.00}{##1}}}
\expandafter\def\csname PY@tok@s\endcsname{\def\PY@tc##1{\textcolor[rgb]{0.73,0.13,0.13}{##1}}}
\expandafter\def\csname PY@tok@sd\endcsname{\let\PY@it=\textit\def\PY@tc##1{\textcolor[rgb]{0.73,0.13,0.13}{##1}}}
\expandafter\def\csname PY@tok@si\endcsname{\let\PY@bf=\textbf\def\PY@tc##1{\textcolor[rgb]{0.73,0.40,0.53}{##1}}}
\expandafter\def\csname PY@tok@se\endcsname{\let\PY@bf=\textbf\def\PY@tc##1{\textcolor[rgb]{0.73,0.40,0.13}{##1}}}
\expandafter\def\csname PY@tok@sr\endcsname{\def\PY@tc##1{\textcolor[rgb]{0.73,0.40,0.53}{##1}}}
\expandafter\def\csname PY@tok@ss\endcsname{\def\PY@tc##1{\textcolor[rgb]{0.10,0.09,0.49}{##1}}}
\expandafter\def\csname PY@tok@sx\endcsname{\def\PY@tc##1{\textcolor[rgb]{0.00,0.50,0.00}{##1}}}
\expandafter\def\csname PY@tok@m\endcsname{\def\PY@tc##1{\textcolor[rgb]{0.40,0.40,0.40}{##1}}}
\expandafter\def\csname PY@tok@gh\endcsname{\let\PY@bf=\textbf\def\PY@tc##1{\textcolor[rgb]{0.00,0.00,0.50}{##1}}}
\expandafter\def\csname PY@tok@gu\endcsname{\let\PY@bf=\textbf\def\PY@tc##1{\textcolor[rgb]{0.50,0.00,0.50}{##1}}}
\expandafter\def\csname PY@tok@gd\endcsname{\def\PY@tc##1{\textcolor[rgb]{0.63,0.00,0.00}{##1}}}
\expandafter\def\csname PY@tok@gi\endcsname{\def\PY@tc##1{\textcolor[rgb]{0.00,0.63,0.00}{##1}}}
\expandafter\def\csname PY@tok@gr\endcsname{\def\PY@tc##1{\textcolor[rgb]{1.00,0.00,0.00}{##1}}}
\expandafter\def\csname PY@tok@ge\endcsname{\let\PY@it=\textit}
\expandafter\def\csname PY@tok@gs\endcsname{\let\PY@bf=\textbf}
\expandafter\def\csname PY@tok@gp\endcsname{\let\PY@bf=\textbf\def\PY@tc##1{\textcolor[rgb]{0.00,0.00,0.50}{##1}}}
\expandafter\def\csname PY@tok@go\endcsname{\def\PY@tc##1{\textcolor[rgb]{0.53,0.53,0.53}{##1}}}
\expandafter\def\csname PY@tok@gt\endcsname{\def\PY@tc##1{\textcolor[rgb]{0.00,0.27,0.87}{##1}}}
\expandafter\def\csname PY@tok@err\endcsname{\def\PY@bc##1{\setlength{\fboxsep}{0pt}\fcolorbox[rgb]{1.00,0.00,0.00}{1,1,1}{\strut ##1}}}
\expandafter\def\csname PY@tok@kc\endcsname{\let\PY@bf=\textbf\def\PY@tc##1{\textcolor[rgb]{0.00,0.50,0.00}{##1}}}
\expandafter\def\csname PY@tok@kd\endcsname{\let\PY@bf=\textbf\def\PY@tc##1{\textcolor[rgb]{0.00,0.50,0.00}{##1}}}
\expandafter\def\csname PY@tok@kn\endcsname{\let\PY@bf=\textbf\def\PY@tc##1{\textcolor[rgb]{0.00,0.50,0.00}{##1}}}
\expandafter\def\csname PY@tok@kr\endcsname{\let\PY@bf=\textbf\def\PY@tc##1{\textcolor[rgb]{0.00,0.50,0.00}{##1}}}
\expandafter\def\csname PY@tok@bp\endcsname{\def\PY@tc##1{\textcolor[rgb]{0.00,0.50,0.00}{##1}}}
\expandafter\def\csname PY@tok@fm\endcsname{\def\PY@tc##1{\textcolor[rgb]{0.00,0.00,1.00}{##1}}}
\expandafter\def\csname PY@tok@vc\endcsname{\def\PY@tc##1{\textcolor[rgb]{0.10,0.09,0.49}{##1}}}
\expandafter\def\csname PY@tok@vg\endcsname{\def\PY@tc##1{\textcolor[rgb]{0.10,0.09,0.49}{##1}}}
\expandafter\def\csname PY@tok@vi\endcsname{\def\PY@tc##1{\textcolor[rgb]{0.10,0.09,0.49}{##1}}}
\expandafter\def\csname PY@tok@vm\endcsname{\def\PY@tc##1{\textcolor[rgb]{0.10,0.09,0.49}{##1}}}
\expandafter\def\csname PY@tok@sa\endcsname{\def\PY@tc##1{\textcolor[rgb]{0.73,0.13,0.13}{##1}}}
\expandafter\def\csname PY@tok@sb\endcsname{\def\PY@tc##1{\textcolor[rgb]{0.73,0.13,0.13}{##1}}}
\expandafter\def\csname PY@tok@sc\endcsname{\def\PY@tc##1{\textcolor[rgb]{0.73,0.13,0.13}{##1}}}
\expandafter\def\csname PY@tok@dl\endcsname{\def\PY@tc##1{\textcolor[rgb]{0.73,0.13,0.13}{##1}}}
\expandafter\def\csname PY@tok@s2\endcsname{\def\PY@tc##1{\textcolor[rgb]{0.73,0.13,0.13}{##1}}}
\expandafter\def\csname PY@tok@sh\endcsname{\def\PY@tc##1{\textcolor[rgb]{0.73,0.13,0.13}{##1}}}
\expandafter\def\csname PY@tok@s1\endcsname{\def\PY@tc##1{\textcolor[rgb]{0.73,0.13,0.13}{##1}}}
\expandafter\def\csname PY@tok@mb\endcsname{\def\PY@tc##1{\textcolor[rgb]{0.40,0.40,0.40}{##1}}}
\expandafter\def\csname PY@tok@mf\endcsname{\def\PY@tc##1{\textcolor[rgb]{0.40,0.40,0.40}{##1}}}
\expandafter\def\csname PY@tok@mh\endcsname{\def\PY@tc##1{\textcolor[rgb]{0.40,0.40,0.40}{##1}}}
\expandafter\def\csname PY@tok@mi\endcsname{\def\PY@tc##1{\textcolor[rgb]{0.40,0.40,0.40}{##1}}}
\expandafter\def\csname PY@tok@il\endcsname{\def\PY@tc##1{\textcolor[rgb]{0.40,0.40,0.40}{##1}}}
\expandafter\def\csname PY@tok@mo\endcsname{\def\PY@tc##1{\textcolor[rgb]{0.40,0.40,0.40}{##1}}}
\expandafter\def\csname PY@tok@ch\endcsname{\let\PY@it=\textit\def\PY@tc##1{\textcolor[rgb]{0.25,0.50,0.50}{##1}}}
\expandafter\def\csname PY@tok@cm\endcsname{\let\PY@it=\textit\def\PY@tc##1{\textcolor[rgb]{0.25,0.50,0.50}{##1}}}
\expandafter\def\csname PY@tok@cpf\endcsname{\let\PY@it=\textit\def\PY@tc##1{\textcolor[rgb]{0.25,0.50,0.50}{##1}}}
\expandafter\def\csname PY@tok@c1\endcsname{\let\PY@it=\textit\def\PY@tc##1{\textcolor[rgb]{0.25,0.50,0.50}{##1}}}
\expandafter\def\csname PY@tok@cs\endcsname{\let\PY@it=\textit\def\PY@tc##1{\textcolor[rgb]{0.25,0.50,0.50}{##1}}}

\def\PYZbs{\char`\\}
\def\PYZus{\char`\_}
\def\PYZob{\char`\{}
\def\PYZcb{\char`\}}
\def\PYZca{\char`\^}
\def\PYZam{\char`\&}
\def\PYZlt{\char`\<}
\def\PYZgt{\char`\>}
\def\PYZsh{\char`\#}
\def\PYZpc{\char`\%}
\def\PYZdl{\char`\$}
\def\PYZhy{\char`\-}
\def\PYZsq{\char`\'}
\def\PYZdq{\char`\"}
\def\PYZti{\char`\~}
% for compatibility with earlier versions
\def\PYZat{@}
\def\PYZlb{[}
\def\PYZrb{]}
\makeatother


    % Exact colors from NB
    \definecolor{incolor}{rgb}{0.0, 0.0, 0.5}
    \definecolor{outcolor}{rgb}{0.545, 0.0, 0.0}



    
    % Prevent overflowing lines due to hard-to-break entities
    \sloppy 
    % Setup hyperref package
    \hypersetup{
      breaklinks=true,  % so long urls are correctly broken across lines
      colorlinks=true,
      urlcolor=urlcolor,
      linkcolor=linkcolor,
      citecolor=citecolor,
      }
    % Slightly bigger margins than the latex defaults
    
    \geometry{verbose,tmargin=1in,bmargin=1in,lmargin=1in,rmargin=1in}
    
    

    \begin{document}
    
    
    \maketitle
    
    

    
    \section{Basis for grayscale images}\label{basis-for-grayscale-images}

\subsection{Introduction}\label{introduction}

Consider the set of real-valued matrices of size \(M\times N\); we can
turn this into a vector space by defining addition and scalar
multiplication in the usual way:

\begin{align}
\mathbf{A} + \mathbf{B} &=  
    \left[ 
        \begin{array}{ccc} 
            a_{0,0} & \dots & a_{0,N-1} \\ 
            \vdots & & \vdots \\ 
            a_{M-1,0} & \dots & b_{M-1,N-1} 
        \end{array}
    \right]
    + 
    \left[ 
        \begin{array}{ccc} 
            b_{0,0} & \dots & b_{0,N-1} \\ 
            \vdots & & \vdots \\ 
            b_{M-1,0} & \dots & b_{M-1,N-1} 
        \end{array}
    \right]
    \\
    &=
    \left[ 
        \begin{array}{ccc} 
            a_{0,0}+b_{0,0} & \dots & a_{0,N-1}+b_{0,N-1} \\ 
            \vdots & & \vdots \\ 
            a_{M-1,0}+b_{M-1,0} & \dots & a_{M-1,N-1}+b_{M-1,N-1} 
        \end{array}
    \right]     
    \\ \\ \\
\beta\mathbf{A} &=  
    \left[ 
        \begin{array}{ccc} 
            \beta a_{0,0} & \dots & \beta a_{0,N-1} \\ 
            \vdots & & \vdots \\ 
            \beta a_{M-1,0} & \dots & \beta a_{M-1,N-1}
        \end{array}
    \right]
\end{align}

As a matter of fact, the space of real-valued \(M\times N\) matrices is
completely equivalent to \(\mathbb{R}^{MN}\) and we can always "unroll"
a matrix into a vector. Assume we proceed column by column; then the
matrix becomes

\[
    \mathbf{a} = \mathbf{A}[:] = [
        \begin{array}{ccccccc}
            a_{0,0} & \dots & a_{M-1,0} & a_{0,1} & \dots & a_{M-1,1} & \ldots & a_{0, N-1} & \dots & a_{M-1,N-1}
        \end{array}]^T
\]

Although the matrix and vector forms represent exactly the same data,
the matrix form allows us to display the data in the form of an image.
Assume each value in the matrix is a grayscale intensity, where zero is
black and 255 is white; for example we can create a checkerboard pattern
of any size with the following function:

    \begin{Verbatim}[commandchars=\\\{\}]
{\color{incolor}In [{\color{incolor}1}]:} \PY{c+c1}{\PYZsh{} usual pyton bookkeeping...}
        \PY{o}{\PYZpc{}}\PY{k}{pylab} inline
        \PY{k+kn}{import} \PY{n+nn}{matplotlib}
        \PY{k+kn}{import} \PY{n+nn}{matplotlib}\PY{n+nn}{.}\PY{n+nn}{pyplot} \PY{k}{as} \PY{n+nn}{plt}
        \PY{k+kn}{import} \PY{n+nn}{numpy} \PY{k}{as} \PY{n+nn}{np}
        \PY{k+kn}{import} \PY{n+nn}{IPython}
        \PY{k+kn}{from} \PY{n+nn}{IPython}\PY{n+nn}{.}\PY{n+nn}{display} \PY{k}{import} \PY{n}{Image}
        \PY{k+kn}{import} \PY{n+nn}{math}
        \PY{k+kn}{from} \PY{n+nn}{\PYZus{}\PYZus{}future\PYZus{}\PYZus{}} \PY{k}{import} \PY{n}{print\PYZus{}function}
        
        \PY{c+c1}{\PYZsh{} ensure all images will be grayscale}
        \PY{n}{gray}\PY{p}{(}\PY{p}{)}\PY{p}{;}
\end{Verbatim}


    \begin{Verbatim}[commandchars=\\\{\}]
/opt/conda/envs/python2/lib/python2.7/site-packages/matplotlib/font\_manager.py:273: UserWarning: Matplotlib is building the font cache using fc-list. This may take a moment.
  warnings.warn('Matplotlib is building the font cache using fc-list. This may take a moment.')

    \end{Verbatim}

    \begin{Verbatim}[commandchars=\\\{\}]
Populating the interactive namespace from numpy and matplotlib

    \end{Verbatim}

    
    \begin{verbatim}
<matplotlib.figure.Figure at 0x7f56f0a85f90>
    \end{verbatim}

    
    \begin{Verbatim}[commandchars=\\\{\}]
{\color{incolor}In [{\color{incolor}2}]:} \PY{c+c1}{\PYZsh{} let\PYZsq{}s create a checkerboard pattern}
        \PY{n}{SIZE} \PY{o}{=} \PY{l+m+mi}{4}
        \PY{n}{img} \PY{o}{=} \PY{n}{np}\PY{o}{.}\PY{n}{zeros}\PY{p}{(}\PY{p}{(}\PY{n}{SIZE}\PY{p}{,} \PY{n}{SIZE}\PY{p}{)}\PY{p}{)}
        \PY{k}{for} \PY{n}{n} \PY{o+ow}{in} \PY{n+nb}{range}\PY{p}{(}\PY{l+m+mi}{0}\PY{p}{,} \PY{n}{SIZE}\PY{p}{)}\PY{p}{:}
            \PY{k}{for} \PY{n}{m} \PY{o+ow}{in} \PY{n+nb}{range}\PY{p}{(}\PY{l+m+mi}{0}\PY{p}{,} \PY{n}{SIZE}\PY{p}{)}\PY{p}{:}
                \PY{k}{if} \PY{p}{(}\PY{n}{n} \PY{o}{\PYZam{}} \PY{l+m+mh}{0x1}\PY{p}{)} \PY{o}{\PYZca{}} \PY{p}{(}\PY{n}{m} \PY{o}{\PYZam{}} \PY{l+m+mh}{0x1}\PY{p}{)}\PY{p}{:}
                    \PY{n}{img}\PY{p}{[}\PY{n}{n}\PY{p}{,} \PY{n}{m}\PY{p}{]} \PY{o}{=} \PY{l+m+mi}{255}
        
        \PY{c+c1}{\PYZsh{} now display the matrix as an image}
        \PY{n}{plt}\PY{o}{.}\PY{n}{matshow}\PY{p}{(}\PY{n}{img}\PY{p}{)}\PY{p}{;} 
\end{Verbatim}


    \begin{center}
    \adjustimage{max size={0.9\linewidth}{0.9\paperheight}}{output_2_0.png}
    \end{center}
    { \hspace*{\fill} \\}
    
    Given the equivalence between the space of \(M\times N\) matrices and
\(\mathbb{R}^{MN}\) we can easily define the inner product between two
matrices in the usual way:

\[
\langle \mathbf{A}, \mathbf{B} \rangle = \sum_{m=0}^{M-1} \sum_{n=0}^{N-1} a_{m,n} b_{m, n}
\]

(where we have neglected the conjugation since we'll only deal with
real-valued matrices); in other words, we can take the inner product
between two matrices as the standard inner product of their unrolled
versions. The inner product allows us to define orthogonality between
images and this is rather useful since we're going to explore a couple
of bases for this space.

\subsection{Actual images}\label{actual-images}

Conveniently, using IPython, we can read images from disk in any given
format and convert them to numpy arrays; let's load and display for
instance a JPEG image:

    \begin{Verbatim}[commandchars=\\\{\}]
{\color{incolor}In [{\color{incolor}3}]:} \PY{n}{img} \PY{o}{=} \PY{n}{np}\PY{o}{.}\PY{n}{array}\PY{p}{(}\PY{n}{plt}\PY{o}{.}\PY{n}{imread}\PY{p}{(}\PY{l+s+s1}{\PYZsq{}}\PY{l+s+s1}{cameraman.jpg}\PY{l+s+s1}{\PYZsq{}}\PY{p}{)}\PY{p}{,} \PY{n}{dtype}\PY{o}{=}\PY{n+nb}{int}\PY{p}{)}
        \PY{n}{plt}\PY{o}{.}\PY{n}{matshow}\PY{p}{(}\PY{n}{img}\PY{p}{)}\PY{p}{;}
\end{Verbatim}


    \begin{center}
    \adjustimage{max size={0.9\linewidth}{0.9\paperheight}}{output_4_0.png}
    \end{center}
    { \hspace*{\fill} \\}
    
    The image is a \(64\times 64\) low-resolution version of the famous
"cameraman" test picture. Out of curiosity, we can look at the first
column of this image, which is is a \(64×1\) vector:

    \begin{Verbatim}[commandchars=\\\{\}]
{\color{incolor}In [{\color{incolor}4}]:} \PY{n}{img}\PY{p}{[}\PY{p}{:}\PY{p}{,}\PY{l+m+mi}{0}\PY{p}{]}
\end{Verbatim}


\begin{Verbatim}[commandchars=\\\{\}]
{\color{outcolor}Out[{\color{outcolor}4}]:} array([156, 157, 157, 152, 154, 155, 151, 157, 152, 155, 158, 159, 159,
               160, 160, 161, 155, 160, 161, 161, 164, 162, 160, 162, 158, 160,
               158, 157, 160, 160, 159, 158, 163, 162, 162, 157, 160, 114, 114,
               103,  88,  62, 109,  82, 108, 128, 138, 140, 136, 128, 122, 137,
               147, 114, 114, 144, 112, 115, 117, 131, 112, 141,  99,  97])
\end{Verbatim}
            
    The values are integers between zero and 255, meaning that each pixel is
encoded over 8 bits (or 256 gray levels).

    \subsection{The canonical basis}\label{the-canonical-basis}

The canonical basis for any matrix space \(\mathbb{R}^{M\times N}\) is
the set of "delta" matrices where only one element equals to one while
all the others are 0. Let's call them \(\mathbf{E}_n\) with
\(0 \leq n < MN\). Here is a function to create the canonical basis
vector given its index:

    \begin{Verbatim}[commandchars=\\\{\}]
{\color{incolor}In [{\color{incolor}5}]:} \PY{k}{def} \PY{n+nf}{canonical}\PY{p}{(}\PY{n}{n}\PY{p}{,} \PY{n}{M}\PY{o}{=}\PY{l+m+mi}{5}\PY{p}{,} \PY{n}{N}\PY{o}{=}\PY{l+m+mi}{10}\PY{p}{)}\PY{p}{:}
            \PY{n}{e} \PY{o}{=} \PY{n}{np}\PY{o}{.}\PY{n}{zeros}\PY{p}{(}\PY{p}{(}\PY{n}{M}\PY{p}{,} \PY{n}{N}\PY{p}{)}\PY{p}{)}
            \PY{n}{e}\PY{p}{[}\PY{p}{(}\PY{n}{n} \PY{o}{\PYZpc{}} \PY{n}{M}\PY{p}{)}\PY{p}{,} \PY{n+nb}{int}\PY{p}{(}\PY{n}{n} \PY{o}{/} \PY{n}{M}\PY{p}{)}\PY{p}{]} \PY{o}{=} \PY{l+m+mi}{1}
            \PY{k}{return} \PY{n}{e}
\end{Verbatim}


    Here are some basis vectors: look for the position of white pixel, which
differentiates them and note that we enumerate pixels column-wise:

    \begin{Verbatim}[commandchars=\\\{\}]
{\color{incolor}In [{\color{incolor}6}]:} \PY{n}{plt}\PY{o}{.}\PY{n}{matshow}\PY{p}{(}\PY{n}{canonical}\PY{p}{(}\PY{l+m+mi}{0}\PY{p}{)}\PY{p}{)}\PY{p}{;}
        \PY{n}{plt}\PY{o}{.}\PY{n}{matshow}\PY{p}{(}\PY{n}{canonical}\PY{p}{(}\PY{l+m+mi}{1}\PY{p}{)}\PY{p}{)}\PY{p}{;}
        \PY{n}{plt}\PY{o}{.}\PY{n}{matshow}\PY{p}{(}\PY{n}{canonical}\PY{p}{(}\PY{l+m+mi}{49}\PY{p}{)}\PY{p}{)}\PY{p}{;}
\end{Verbatim}


    \begin{center}
    \adjustimage{max size={0.9\linewidth}{0.9\paperheight}}{output_11_0.png}
    \end{center}
    { \hspace*{\fill} \\}
    
    \begin{center}
    \adjustimage{max size={0.9\linewidth}{0.9\paperheight}}{output_11_1.png}
    \end{center}
    { \hspace*{\fill} \\}
    
    \begin{center}
    \adjustimage{max size={0.9\linewidth}{0.9\paperheight}}{output_11_2.png}
    \end{center}
    { \hspace*{\fill} \\}
    
    \subsection{Transmitting images}\label{transmitting-images}

Suppose we want to transmit the "cameraman" image over a communication
channel. The intuitive way to do so is to send the pixel values one by
one, which corresponds to sending the coefficients of the decomposition
of the image over the canonical basis. So far, nothing complicated: to
send the cameraman image, for instance, we will send
\(64\times 64 = 4096\) coefficients in a row.

Now suppose that a communication failure takes place after the first
half of the pixels have been sent. The received data will allow us to
display an approximation of the original image only. If we replace the
missing data with zeros, here is what we would see, which is not very
pretty:

    \begin{Verbatim}[commandchars=\\\{\}]
{\color{incolor}In [{\color{incolor}7}]:} \PY{c+c1}{\PYZsh{} unrolling of the image for transmission (we go column by column, hence \PYZdq{}F\PYZdq{})}
        \PY{n}{tx\PYZus{}img} \PY{o}{=} \PY{n}{np}\PY{o}{.}\PY{n}{ravel}\PY{p}{(}\PY{n}{img}\PY{p}{,} \PY{l+s+s2}{\PYZdq{}}\PY{l+s+s2}{F}\PY{l+s+s2}{\PYZdq{}}\PY{p}{)}
        
        \PY{c+c1}{\PYZsh{} oops, we lose half the data}
        \PY{n}{tx\PYZus{}img}\PY{p}{[}\PY{n+nb}{int}\PY{p}{(}\PY{n+nb}{len}\PY{p}{(}\PY{n}{tx\PYZus{}img}\PY{p}{)}\PY{o}{/}\PY{l+m+mi}{2}\PY{p}{)}\PY{p}{:}\PY{p}{]} \PY{o}{=} \PY{l+m+mi}{0}
        
        \PY{c+c1}{\PYZsh{} rebuild matrix}
        \PY{n}{rx\PYZus{}img} \PY{o}{=} \PY{n}{np}\PY{o}{.}\PY{n}{reshape}\PY{p}{(}\PY{n}{tx\PYZus{}img}\PY{p}{,} \PY{p}{(}\PY{l+m+mi}{64}\PY{p}{,} \PY{l+m+mi}{64}\PY{p}{)}\PY{p}{,} \PY{l+s+s2}{\PYZdq{}}\PY{l+s+s2}{F}\PY{l+s+s2}{\PYZdq{}}\PY{p}{)}
        \PY{n}{plt}\PY{o}{.}\PY{n}{matshow}\PY{p}{(}\PY{n}{rx\PYZus{}img}\PY{p}{)}\PY{p}{;}
\end{Verbatim}


    \begin{center}
    \adjustimage{max size={0.9\linewidth}{0.9\paperheight}}{output_13_0.png}
    \end{center}
    { \hspace*{\fill} \\}
    
    Can we come up with a trasmission scheme that is more robust in the face
of channel loss? Interestingly, the answer is yes, and it involves a
different, more versatile basis for the space of images. What we will do
is the following:

\begin{itemize}
\tightlist
\item
  describe the Haar basis, a new basis for the image space
\item
  project the image in the new basis
\item
  transmit the projection coefficients
\item
  rebuild the image using the basis vectors
\end{itemize}

We know a few things: if we choose an orthonormal basis, the analysis
and synthesis formulas will be super easy (a simple inner product and a
scalar multiplication respectively). The trick is to find a basis that
will be robust to the loss of some coefficients.

One such basis is the \textbf{Haar basis}. We cannot go into too many
details in this notebook but, for the curious, a good starting point is
\href{https://chengtsolin.wordpress.com/2015/04/15/real-time-2d-discrete-wavelet-transform-using-opengl-compute-shader/}{here}.
Mathematical formulas aside, the Haar basis works by encoding the
information in a \emph{hierarchical} way: the first basis vectors encode
the broad information and the higher coefficients encode the detail.
Let's have a look.

First of all, to keep things simple, we will remain in the space of
square matrices whose size is a power of two. The code to generate the
Haar basis matrices is the following: first we generate a 1D Haar vector
and then we obtain the basis matrices by taking the outer product of all
possible 1D vectors (don't worry if it's not clear, the results are
what's important):

    \begin{Verbatim}[commandchars=\\\{\}]
{\color{incolor}In [{\color{incolor}8}]:} \PY{k}{def} \PY{n+nf}{haar1D}\PY{p}{(}\PY{n}{n}\PY{p}{,} \PY{n}{SIZE}\PY{p}{)}\PY{p}{:}
            \PY{c+c1}{\PYZsh{} check power of two}
            \PY{k}{if} \PY{n}{math}\PY{o}{.}\PY{n}{floor}\PY{p}{(}\PY{n}{math}\PY{o}{.}\PY{n}{log}\PY{p}{(}\PY{n}{SIZE}\PY{p}{)} \PY{o}{/} \PY{n}{math}\PY{o}{.}\PY{n}{log}\PY{p}{(}\PY{l+m+mi}{2}\PY{p}{)}\PY{p}{)} \PY{o}{!=} \PY{n}{math}\PY{o}{.}\PY{n}{log}\PY{p}{(}\PY{n}{SIZE}\PY{p}{)} \PY{o}{/} \PY{n}{math}\PY{o}{.}\PY{n}{log}\PY{p}{(}\PY{l+m+mi}{2}\PY{p}{)}\PY{p}{:}
                \PY{n+nb}{print}\PY{p}{(}\PY{l+s+s2}{\PYZdq{}}\PY{l+s+s2}{Haar defined only for lengths that are a power of two}\PY{l+s+s2}{\PYZdq{}}\PY{p}{)}
                \PY{k}{return} \PY{k+kc}{None}
            \PY{k}{if} \PY{n}{n} \PY{o}{\PYZgt{}}\PY{o}{=} \PY{n}{SIZE} \PY{o+ow}{or} \PY{n}{n} \PY{o}{\PYZlt{}} \PY{l+m+mi}{0}\PY{p}{:}
                \PY{n+nb}{print}\PY{p}{(}\PY{l+s+s2}{\PYZdq{}}\PY{l+s+s2}{invalid Haar index}\PY{l+s+s2}{\PYZdq{}}\PY{p}{)}
                \PY{k}{return} \PY{k+kc}{None}
            
            \PY{c+c1}{\PYZsh{} zero basis vector}
            \PY{k}{if} \PY{n}{n} \PY{o}{==} \PY{l+m+mi}{0}\PY{p}{:}
                \PY{k}{return} \PY{n}{np}\PY{o}{.}\PY{n}{ones}\PY{p}{(}\PY{n}{SIZE}\PY{p}{)}
            
            \PY{c+c1}{\PYZsh{} express n \PYZgt{} 1 as 2\PYZca{}p + q with p as large as possible;}
            \PY{c+c1}{\PYZsh{} then k = SIZE/2\PYZca{}p is the length of the support}
            \PY{c+c1}{\PYZsh{} and s = qk is the shift}
            \PY{n}{p} \PY{o}{=} \PY{n}{math}\PY{o}{.}\PY{n}{floor}\PY{p}{(}\PY{n}{math}\PY{o}{.}\PY{n}{log}\PY{p}{(}\PY{n}{n}\PY{p}{)} \PY{o}{/} \PY{n}{math}\PY{o}{.}\PY{n}{log}\PY{p}{(}\PY{l+m+mi}{2}\PY{p}{)}\PY{p}{)}
            \PY{n}{pp} \PY{o}{=} \PY{n+nb}{int}\PY{p}{(}\PY{n+nb}{pow}\PY{p}{(}\PY{l+m+mi}{2}\PY{p}{,} \PY{n}{p}\PY{p}{)}\PY{p}{)}
            \PY{n}{k} \PY{o}{=} \PY{n}{SIZE} \PY{o}{/} \PY{n}{pp}
            \PY{n}{s} \PY{o}{=} \PY{p}{(}\PY{n}{n} \PY{o}{\PYZhy{}} \PY{n}{pp}\PY{p}{)} \PY{o}{*} \PY{n}{k}
            
            \PY{n}{h} \PY{o}{=} \PY{n}{np}\PY{o}{.}\PY{n}{zeros}\PY{p}{(}\PY{n}{SIZE}\PY{p}{)}
            \PY{n}{h}\PY{p}{[}\PY{n+nb}{int}\PY{p}{(}\PY{n}{s}\PY{p}{)}\PY{p}{:}\PY{n+nb}{int}\PY{p}{(}\PY{n}{s}\PY{o}{+}\PY{n}{k}\PY{o}{/}\PY{l+m+mi}{2}\PY{p}{)}\PY{p}{]} \PY{o}{=} \PY{l+m+mi}{1}
            \PY{n}{h}\PY{p}{[}\PY{n+nb}{int}\PY{p}{(}\PY{n}{s}\PY{o}{+}\PY{n}{k}\PY{o}{/}\PY{l+m+mi}{2}\PY{p}{)}\PY{p}{:}\PY{n+nb}{int}\PY{p}{(}\PY{n}{s}\PY{o}{+}\PY{n}{k}\PY{p}{)}\PY{p}{]} \PY{o}{=} \PY{o}{\PYZhy{}}\PY{l+m+mi}{1}
            \PY{c+c1}{\PYZsh{} these are not normalized}
            \PY{k}{return} \PY{n}{h}
        
        
        \PY{k}{def} \PY{n+nf}{haar2D}\PY{p}{(}\PY{n}{n}\PY{p}{,} \PY{n}{SIZE}\PY{o}{=}\PY{l+m+mi}{8}\PY{p}{)}\PY{p}{:}
            \PY{c+c1}{\PYZsh{} get horizontal and vertical indices}
            \PY{n}{hr} \PY{o}{=} \PY{n}{haar1D}\PY{p}{(}\PY{n}{n} \PY{o}{\PYZpc{}} \PY{n}{SIZE}\PY{p}{,} \PY{n}{SIZE}\PY{p}{)}
            \PY{n}{hv} \PY{o}{=} \PY{n}{haar1D}\PY{p}{(}\PY{n+nb}{int}\PY{p}{(}\PY{n}{n} \PY{o}{/} \PY{n}{SIZE}\PY{p}{)}\PY{p}{,} \PY{n}{SIZE}\PY{p}{)}
            \PY{c+c1}{\PYZsh{} 2D Haar basis matrix is separable, so we can}
            \PY{c+c1}{\PYZsh{}  just take the column\PYZhy{}row product}
            \PY{n}{H} \PY{o}{=} \PY{n}{np}\PY{o}{.}\PY{n}{outer}\PY{p}{(}\PY{n}{hr}\PY{p}{,} \PY{n}{hv}\PY{p}{)}
            \PY{n}{H} \PY{o}{=} \PY{n}{H} \PY{o}{/} \PY{n}{math}\PY{o}{.}\PY{n}{sqrt}\PY{p}{(}\PY{n}{np}\PY{o}{.}\PY{n}{sum}\PY{p}{(}\PY{n}{H} \PY{o}{*} \PY{n}{H}\PY{p}{)}\PY{p}{)}
            \PY{k}{return} \PY{n}{H}
\end{Verbatim}


    First of all, let's look at a few basis matrices; note that the matrices
have positive and negative values, so that the value of zero will be
represented as gray:

    \begin{Verbatim}[commandchars=\\\{\}]
{\color{incolor}In [{\color{incolor}12}]:} \PY{n}{plt}\PY{o}{.}\PY{n}{matshow}\PY{p}{(}\PY{n}{haar2D}\PY{p}{(}\PY{l+m+mi}{0}\PY{p}{)}\PY{p}{)}\PY{p}{;}
         \PY{n}{plt}\PY{o}{.}\PY{n}{matshow}\PY{p}{(}\PY{n}{haar2D}\PY{p}{(}\PY{l+m+mi}{1}\PY{p}{)}\PY{p}{)}\PY{p}{;}
         \PY{n}{plt}\PY{o}{.}\PY{n}{matshow}\PY{p}{(}\PY{n}{haar2D}\PY{p}{(}\PY{l+m+mi}{10}\PY{p}{)}\PY{p}{)}\PY{p}{;}
         \PY{n}{plt}\PY{o}{.}\PY{n}{matshow}\PY{p}{(}\PY{n}{haar2D}\PY{p}{(}\PY{l+m+mi}{63}\PY{p}{)}\PY{p}{)}\PY{p}{;}
\end{Verbatim}


    \begin{center}
    \adjustimage{max size={0.9\linewidth}{0.9\paperheight}}{output_17_0.png}
    \end{center}
    { \hspace*{\fill} \\}
    
    \begin{center}
    \adjustimage{max size={0.9\linewidth}{0.9\paperheight}}{output_17_1.png}
    \end{center}
    { \hspace*{\fill} \\}
    
    \begin{center}
    \adjustimage{max size={0.9\linewidth}{0.9\paperheight}}{output_17_2.png}
    \end{center}
    { \hspace*{\fill} \\}
    
    \begin{center}
    \adjustimage{max size={0.9\linewidth}{0.9\paperheight}}{output_17_3.png}
    \end{center}
    { \hspace*{\fill} \\}
    
    We can notice two key properties

\begin{itemize}
\tightlist
\item
  each basis matrix has positive and negative values in some symmetric
  patter: this means that the basis matrix will implicitly compute the
  difference between image areas
\item
  low-index basis matrices take differences between large areas, while
  high-index ones take differences in smaller \textbf{localized} areas
  of the image
\end{itemize}

We can immediately verify that the Haar matrices are orthogonal:

    \begin{Verbatim}[commandchars=\\\{\}]
{\color{incolor}In [{\color{incolor}13}]:} \PY{c+c1}{\PYZsh{} let\PYZsq{}s use an 8x8 space; there will be 64 basis vectors}
         \PY{c+c1}{\PYZsh{} compute all possible inner product and only print the nonzero results}
         \PY{k}{for} \PY{n}{m} \PY{o+ow}{in} \PY{n+nb}{range}\PY{p}{(}\PY{l+m+mi}{0}\PY{p}{,}\PY{l+m+mi}{64}\PY{p}{)}\PY{p}{:}
             \PY{k}{for} \PY{n}{n} \PY{o+ow}{in} \PY{n+nb}{range}\PY{p}{(}\PY{l+m+mi}{0}\PY{p}{,}\PY{l+m+mi}{64}\PY{p}{)}\PY{p}{:}
                 \PY{n}{r} \PY{o}{=} \PY{n}{np}\PY{o}{.}\PY{n}{sum}\PY{p}{(}\PY{n}{haar2D}\PY{p}{(}\PY{n}{m}\PY{p}{,} \PY{l+m+mi}{8}\PY{p}{)} \PY{o}{*} \PY{n}{haar2D}\PY{p}{(}\PY{n}{n}\PY{p}{,} \PY{l+m+mi}{8}\PY{p}{)}\PY{p}{)}
                 \PY{k}{if} \PY{n}{r} \PY{o}{!=} \PY{l+m+mi}{0}\PY{p}{:}
                     \PY{n+nb}{print}\PY{p}{(}\PY{l+s+s2}{\PYZdq{}}\PY{l+s+s2}{[}\PY{l+s+si}{\PYZpc{}d}\PY{l+s+s2}{x}\PY{l+s+si}{\PYZpc{}d}\PY{l+s+s2}{ \PYZhy{}\PYZgt{} }\PY{l+s+si}{\PYZpc{}f}\PY{l+s+s2}{] }\PY{l+s+s2}{\PYZdq{}} \PY{o}{\PYZpc{}} \PY{p}{(}\PY{n}{m}\PY{p}{,} \PY{n}{n}\PY{p}{,} \PY{n}{r}\PY{p}{)}\PY{p}{,} \PY{n}{end}\PY{o}{=}\PY{l+s+s2}{\PYZdq{}}\PY{l+s+s2}{\PYZdq{}}\PY{p}{)}
\end{Verbatim}


    \begin{Verbatim}[commandchars=\\\{\}]
[0x0 -> 1.000000] [1x1 -> 1.000000] [2x2 -> 1.000000] [3x3 -> 1.000000] [4x4 -> 1.000000] [5x5 -> 1.000000] [6x6 -> 1.000000] [7x7 -> 1.000000] [8x8 -> 1.000000] [9x9 -> 1.000000] [10x10 -> 1.000000] [11x11 -> 1.000000] [12x12 -> 1.000000] [13x13 -> 1.000000] [14x14 -> 1.000000] [15x15 -> 1.000000] [16x16 -> 1.000000] [16x17 -> -0.000000] [17x16 -> -0.000000] [17x17 -> 1.000000] [18x18 -> 1.000000] [19x19 -> 1.000000] [20x20 -> 1.000000] [21x21 -> 1.000000] [22x22 -> 1.000000] [23x23 -> 1.000000] [24x24 -> 1.000000] [24x25 -> -0.000000] [25x24 -> -0.000000] [25x25 -> 1.000000] [26x26 -> 1.000000] [27x27 -> 1.000000] [28x28 -> 1.000000] [29x29 -> 1.000000] [30x30 -> 1.000000] [31x31 -> 1.000000] [32x32 -> 1.000000] [33x33 -> 1.000000] [34x34 -> 1.000000] [35x35 -> 1.000000] [36x36 -> 1.000000] [37x37 -> 1.000000] [38x38 -> 1.000000] [39x39 -> 1.000000] [40x40 -> 1.000000] [41x41 -> 1.000000] [42x42 -> 1.000000] [43x43 -> 1.000000] [44x44 -> 1.000000] [45x45 -> 1.000000] [46x46 -> 1.000000] [47x47 -> 1.000000] [48x48 -> 1.000000] [49x49 -> 1.000000] [50x50 -> 1.000000] [51x51 -> 1.000000] [52x52 -> 1.000000] [53x53 -> 1.000000] [54x54 -> 1.000000] [55x55 -> 1.000000] [56x56 -> 1.000000] [57x57 -> 1.000000] [58x58 -> 1.000000] [59x59 -> 1.000000] [60x60 -> 1.000000] [61x61 -> 1.000000] [62x62 -> 1.000000] [63x63 -> 1.000000] 
    \end{Verbatim}

    OK! Everything's fine. Now let's transmit the "cameraman" image: first,
let's verify that it works

    \begin{Verbatim}[commandchars=\\\{\}]
{\color{incolor}In [{\color{incolor}14}]:} \PY{c+c1}{\PYZsh{} project the image onto the Haar basis, obtaining a vector of 4096 coefficients}
         \PY{c+c1}{\PYZsh{} this is simply the analysis formula for the vector space with an orthogonal basis}
         \PY{n}{tx\PYZus{}img} \PY{o}{=} \PY{n}{np}\PY{o}{.}\PY{n}{zeros}\PY{p}{(}\PY{l+m+mi}{64}\PY{o}{*}\PY{l+m+mi}{64}\PY{p}{)}
         \PY{k}{for} \PY{n}{k} \PY{o+ow}{in} \PY{n+nb}{range}\PY{p}{(}\PY{l+m+mi}{0}\PY{p}{,} \PY{p}{(}\PY{l+m+mi}{64}\PY{o}{*}\PY{l+m+mi}{64}\PY{p}{)}\PY{p}{)}\PY{p}{:}
             \PY{n}{tx\PYZus{}img}\PY{p}{[}\PY{n}{k}\PY{p}{]} \PY{o}{=} \PY{n}{np}\PY{o}{.}\PY{n}{sum}\PY{p}{(}\PY{n}{img} \PY{o}{*} \PY{n}{haar2D}\PY{p}{(}\PY{n}{k}\PY{p}{,} \PY{l+m+mi}{64}\PY{p}{)}\PY{p}{)}
         
         \PY{c+c1}{\PYZsh{} now rebuild the image with the synthesis formula; since the basis is orthonormal}
         \PY{c+c1}{\PYZsh{}  we just need to scale the basis matrices by the projection coefficients}
         \PY{n}{rx\PYZus{}img} \PY{o}{=} \PY{n}{np}\PY{o}{.}\PY{n}{zeros}\PY{p}{(}\PY{p}{(}\PY{l+m+mi}{64}\PY{p}{,} \PY{l+m+mi}{64}\PY{p}{)}\PY{p}{)}
         \PY{k}{for} \PY{n}{k} \PY{o+ow}{in} \PY{n+nb}{range}\PY{p}{(}\PY{l+m+mi}{0}\PY{p}{,} \PY{p}{(}\PY{l+m+mi}{64}\PY{o}{*}\PY{l+m+mi}{64}\PY{p}{)}\PY{p}{)}\PY{p}{:}
             \PY{n}{rx\PYZus{}img} \PY{o}{+}\PY{o}{=} \PY{n}{tx\PYZus{}img}\PY{p}{[}\PY{n}{k}\PY{p}{]} \PY{o}{*} \PY{n}{haar2D}\PY{p}{(}\PY{n}{k}\PY{p}{,} \PY{l+m+mi}{64}\PY{p}{)}
         
         \PY{n}{plt}\PY{o}{.}\PY{n}{matshow}\PY{p}{(}\PY{n}{rx\PYZus{}img}\PY{p}{)}\PY{p}{;}
\end{Verbatim}


    \begin{center}
    \adjustimage{max size={0.9\linewidth}{0.9\paperheight}}{output_21_0.png}
    \end{center}
    { \hspace*{\fill} \\}
    
    Cool, it works! Now let's see what happens if we lose the second half of
the coefficients:

    \begin{Verbatim}[commandchars=\\\{\}]
{\color{incolor}In [{\color{incolor}15}]:} \PY{c+c1}{\PYZsh{} oops, we lose half the data}
         \PY{n}{lossy\PYZus{}img} \PY{o}{=} \PY{n}{np}\PY{o}{.}\PY{n}{copy}\PY{p}{(}\PY{n}{tx\PYZus{}img}\PY{p}{)}\PY{p}{;}
         \PY{n}{lossy\PYZus{}img}\PY{p}{[}\PY{n+nb}{int}\PY{p}{(}\PY{n+nb}{len}\PY{p}{(}\PY{n}{tx\PYZus{}img}\PY{p}{)}\PY{o}{/}\PY{l+m+mi}{2}\PY{p}{)}\PY{p}{:}\PY{p}{]} \PY{o}{=} \PY{l+m+mi}{0}
         
         \PY{c+c1}{\PYZsh{} rebuild matrix}
         \PY{n}{rx\PYZus{}img} \PY{o}{=} \PY{n}{np}\PY{o}{.}\PY{n}{zeros}\PY{p}{(}\PY{p}{(}\PY{l+m+mi}{64}\PY{p}{,} \PY{l+m+mi}{64}\PY{p}{)}\PY{p}{)}
         \PY{k}{for} \PY{n}{k} \PY{o+ow}{in} \PY{n+nb}{range}\PY{p}{(}\PY{l+m+mi}{0}\PY{p}{,} \PY{p}{(}\PY{l+m+mi}{64}\PY{o}{*}\PY{l+m+mi}{64}\PY{p}{)}\PY{p}{)}\PY{p}{:}
             \PY{n}{rx\PYZus{}img} \PY{o}{+}\PY{o}{=} \PY{n}{lossy\PYZus{}img}\PY{p}{[}\PY{n}{k}\PY{p}{]} \PY{o}{*} \PY{n}{haar2D}\PY{p}{(}\PY{n}{k}\PY{p}{,} \PY{l+m+mi}{64}\PY{p}{)}
         
         \PY{n}{plt}\PY{o}{.}\PY{n}{matshow}\PY{p}{(}\PY{n}{rx\PYZus{}img}\PY{p}{)}\PY{p}{;}
\end{Verbatim}


    \begin{center}
    \adjustimage{max size={0.9\linewidth}{0.9\paperheight}}{output_23_0.png}
    \end{center}
    { \hspace*{\fill} \\}
    
    That's quite remarkable, no? We've lost the same amount of information
as before but the image is still acceptable. This is because we lost the
coefficients associated to the fine details of the image but we retained
the "broad strokes" encoded by the first half.

Note that if we lose the first half of the coefficients the result is
markedly different:

    \begin{Verbatim}[commandchars=\\\{\}]
{\color{incolor}In [{\color{incolor}16}]:} \PY{n}{lossy\PYZus{}img} \PY{o}{=} \PY{n}{np}\PY{o}{.}\PY{n}{copy}\PY{p}{(}\PY{n}{tx\PYZus{}img}\PY{p}{)}\PY{p}{;}
         \PY{n}{lossy\PYZus{}img}\PY{p}{[}\PY{l+m+mi}{0}\PY{p}{:}\PY{n+nb}{int}\PY{p}{(}\PY{n+nb}{len}\PY{p}{(}\PY{n}{tx\PYZus{}img}\PY{p}{)}\PY{o}{/}\PY{l+m+mi}{2}\PY{p}{)}\PY{p}{]} \PY{o}{=} \PY{l+m+mi}{0}
         
         \PY{n}{rx\PYZus{}img} \PY{o}{=} \PY{n}{np}\PY{o}{.}\PY{n}{zeros}\PY{p}{(}\PY{p}{(}\PY{l+m+mi}{64}\PY{p}{,} \PY{l+m+mi}{64}\PY{p}{)}\PY{p}{)}
         \PY{k}{for} \PY{n}{k} \PY{o+ow}{in} \PY{n+nb}{range}\PY{p}{(}\PY{l+m+mi}{0}\PY{p}{,} \PY{p}{(}\PY{l+m+mi}{64}\PY{o}{*}\PY{l+m+mi}{64}\PY{p}{)}\PY{p}{)}\PY{p}{:}
             \PY{n}{rx\PYZus{}img} \PY{o}{+}\PY{o}{=} \PY{n}{lossy\PYZus{}img}\PY{p}{[}\PY{n}{k}\PY{p}{]} \PY{o}{*} \PY{n}{haar2D}\PY{p}{(}\PY{n}{k}\PY{p}{,} \PY{l+m+mi}{64}\PY{p}{)}
         
         \PY{n}{plt}\PY{o}{.}\PY{n}{matshow}\PY{p}{(}\PY{n}{rx\PYZus{}img}\PY{p}{)}\PY{p}{;}
\end{Verbatim}


    \begin{center}
    \adjustimage{max size={0.9\linewidth}{0.9\paperheight}}{output_25_0.png}
    \end{center}
    { \hspace*{\fill} \\}
    
    In fact, schemes like this one are used in \emph{progressive encoding}:
send the most important information first and add details if the channel
permits it. You may have experienced this while browsing the interned
over a slow connection.

All in all, a great application of a change of basis!


    % Add a bibliography block to the postdoc
    
    
    
    \end{document}
